%%%%%%%%%%%%%%%%%%%%%%%%%%%%%%%%%%%%%%%%%
% Wenneker Assignment
% LaTeX Template
% Version 2.0 (12/1/2019)
%
% This template originates from:
% http://www.LaTeXTemplates.com
%
% Authors:
% Vel (vel@LaTeXTemplates.com)
% Frits Wenneker
%
% License:
% CC BY-NC-SA 3.0 (http://creativecommons.org/licenses/by-nc-sa/3.0/)
% 
%%%%%%%%%%%%%%%%%%%%%%%%%%%%%%%%%%%%%%%%%

%----------------------------------------------------------------------------------------
%	PACKAGES AND OTHER DOCUMENT CONFIGURATIONS
%----------------------------------------------------------------------------------------

\documentclass[14pt]{scrartcl} % Font size

%%%%%%%%%%%%%%%%%%%%%%%%%%%%%%%%%%%%%%%%%
% Wenneker Assignment
% Structure Specification File
% Version 2.0 (12/1/2019)
%
% This template originates from:
% http://www.LaTeXTemplates.com
%
% Authors:
% Vel (vel@LaTeXTemplates.com)
% Frits Wenneker
%
% License:
% CC BY-NC-SA 3.0 (http://creativecommons.org/licenses/by-nc-sa/3.0/)
% 
%%%%%%%%%%%%%%%%%%%%%%%%%%%%%%%%%%%%%%%%%

%----------------------------------------------------------------------------------------
%	PACKAGES AND OTHER DOCUMENT CONFIGURATIONS
%----------------------------------------------------------------------------------------

\usepackage{amsmath, amsfonts, amsthm} % Math packages

\usepackage[utf8]{inputenc}

\usepackage[UTF8]{ctex}

\usepackage{listings} % Code listings, with syntax highlighting

\usepackage[english]{babel} % English language hyphenation

\usepackage{graphicx} % Required for inserting images
\graphicspath{{Figures/}{./}} % Specifies where to look for included images (trailing slash required)

\usepackage{booktabs} % Required for better horizontal rules in tables

\numberwithin{equation}{section} % Number equations within sections (i.e. 1.1, 1.2, 2.1, 2.2 instead of 1, 2, 3, 4)
\numberwithin{figure}{section} % Number figures within sections (i.e. 1.1, 1.2, 2.1, 2.2 instead of 1, 2, 3, 4)
\numberwithin{table}{section} % Number tables within sections (i.e. 1.1, 1.2, 2.1, 2.2 instead of 1, 2, 3, 4)

\setlength\parindent{0pt} % Removes all indentation from paragraphs

\usepackage{enumitem} % Required for list customisation
\setlist{noitemsep} % No spacing between list items

%----------------------------------------------------------------------------------------
%	DOCUMENT MARGINS
%----------------------------------------------------------------------------------------

\usepackage{geometry} % Required for adjusting page dimensions and margins

\geometry{
	paper=a4paper, % Paper size, change to letterpaper for US letter size
	top=2.5cm, % Top margin
	bottom=3cm, % Bottom margin
	left=3cm, % Left margin
	right=3cm, % Right margin
	headheight=0.75cm, % Header height
	footskip=1.5cm, % Space from the bottom margin to the baseline of the footer
	headsep=0.75cm, % Space from the top margin to the baseline of the header
	%showframe, % Uncomment to show how the type block is set on the page
}

%----------------------------------------------------------------------------------------
%	FONTS
%----------------------------------------------------------------------------------------

\usepackage[utf8]{inputenc} % Required for inputting international characters
\usepackage[T1]{fontenc} % Use 8-bit encoding

\usepackage{fourier} % Use the Adobe Utopia font for the document

%----------------------------------------------------------------------------------------
%	SECTION TITLES
%----------------------------------------------------------------------------------------

\usepackage{sectsty} % Allows customising section commands

\sectionfont{\vspace{6pt}\centering\normalfont\scshape} % \section{} styling
\subsectionfont{\normalfont\bfseries} % \subsection{} styling
\subsubsectionfont{\normalfont\itshape} % \subsubsection{} styling
\paragraphfont{\normalfont\scshape} % \paragraph{} styling

%----------------------------------------------------------------------------------------
%	HEADERS AND FOOTERS
%----------------------------------------------------------------------------------------

\usepackage{scrlayer-scrpage} % Required for customising headers and footers

\ohead*{} % Right header
\ihead*{} % Left header
\chead*{} % Centre header

\ofoot*{} % Right footer
\ifoot*{} % Left footer
\cfoot*{\pagemark} % Centre footer
 % Include the file specifying the document structure and custom commands

%----------------------------------------------------------------------------------------
%	TITLE SECTION
%----------------------------------------------------------------------------------------

\title{	
	\normalfont\normalsize
	%\vspace{25pt} % Whitespace
	\rule{\linewidth}{0.5pt}\\ % Thin top horizontal rule
	\vspace{20pt} % Whitespace
	{\huge 实验一	MiniDraw}\\ % The assignment title
	\vspace{12pt} % Whitespace
	\rule{\linewidth}{2pt}\\ % Thick bottom horizontal rule
	\vspace{12pt} % Whitespace
}

\author{\LARGE ID: 58		陈文博} % Your name

\date{\normalsize\today} % Today's date (\today) or a custom date

\begin{document}

\maketitle % Print the title

%----------------------------------------------------------------------------------------
%	FIGURE EXAMPLE
%----------------------------------------------------------------------------------------

\section{实验要求}

%\begin{figure}[h] % [h] forces the figure to be output where it is defined in the code (it suppresses floating)
%	\centering
%	\includegraphics[width=0.5\columnwidth]{swallow.jpg} % Example image
%	\caption{European swallow.}
%\end{figure}

\begin{itemize}
	\item[*] 写一个画图小程序 MiniDraw,要求画直线 (Line),椭圆 (Ellipse),矩形 (Rectangle),多边形 (Polygon) 等图形元素(图元)
	\item[*] 每种图元需用一个类(对象)来封装,如 `CLine`,`CEllipse`,`CRect`,`CPolygon`,`CFreehand` 
	\item[*] 各种图元从一个父类来继承,如 `CFigure` 
	\item[*] 学习类的继承和多态
\end{itemize}

\pagebreak
\section{功能描述}

在实验基本要求基础上,我还增加了一下功能:
\begin{itemize}
	\item 设置画笔颜色(color)、宽度(width)
	\item 闭合图形颜色填充(fill)
	\item 绘制平滑曲线
	\item 撤销(Undo)绘图
	\item 保存(Save)画板
	\item 给工具栏、菜单栏和主窗口设置了图标
\end{itemize}
基本效果如下:

\begin{figure}[h] % [h] forces the figure to be output where it is defined in the code (it suppresses floating)
	\centering
	\includegraphics[width=0.5\columnwidth]{2.png} % Example image
	\caption{实验效果}
\end{figure}


%------------------------------------------------

\section{开发环境}

\textbf{IDE}:Microsoft Visual Studio 2019 community\\
\textbf{CMake}:3.16.3\\
\textbf{Qt}:5.14.1


%----------------------------------------------------------------------------------------
%	TEXT EXAMPLE
%----------------------------------------------------------------------------------------
\pagebreak
\section{架构设计}

\subsection{文件结构}

\begin{figure}[h] % [h] forces the figure to be output where it is defined in the code (it suppresses floating)
	\centering
	\includegraphics[width=0.5\columnwidth]{1.png} % Example image
	\caption{文件结构}
\end{figure}

%------------------------------------------------
\pagebreak
\subsection{各个类的继承关系}

\begin{figure}[h] % [h] forces the figure to be output where it is defined in the code (it suppresses floating)
	\centering
	\includegraphics[width=1.2\columnwidth]{3.png} % Example image
	\caption{各个类的继承关系}
\end{figure}

%------------------------------------------------
\pagebreak
\section{设计难点与解决}

\subsection{Freedraw的实现}

使用Qt自带的类QPainterPath存储Freedraw产生的轨迹点,在绘图刷新阶段使用painter的drawPath方法进行绘制。这种方法比循环调用drawLine方法效率要高得多。
%------------------------------------------------


\subsection{Polygon的实现}

和Freedraw类似,Polygon也需要一个存储多边形各个顶点的结构,Qt同样提供了QPolygon类实现该功能。由于Polygon涉及到两种鼠标操作:左击开始选取顶点(存储点);右击连接闭合图形。为此引入类方法update,在shape父类中添加虚函数update,在Polygon子类中定义,目的是通过鼠标左击右击修改控制变量mode来切换绘点模式和连接闭合模式,实现功能要求。

\subsection{Undo的实现}

画板为实现显示之前所有绘制的图像,维护了一个vector类来存储所有产生的shape子类,要实现undo,只需要将vector中最后一个添加的元素删除即可,这里要注意在pop\_back之前应该先delete最后一个元素,否则会发生内存泄漏。

\begin{figure}[h] % [h] forces the figure to be output where it is defined in the code (it suppresses floating)
	\centering
	\includegraphics[width=0.7\columnwidth]{10.png} % Example image
	\caption{程序截图}
\end{figure}

\pagebreak
\subsection{设置线宽、线色和颜色填充}
绘图的线宽、线色和填充色可以通过对painter进行设置,为了能够保存并显示所有形状的颜色,在父类shape中添加相应的属性line\_color、fill\_color和width,设置相关Action并添加到工具栏中。用对话框qcolordialog和qinputdialog进行颜色选择和线宽输入,在遍历shape\_list\_时读取对象的属性修改painter,达到相应的效果。

\subsection{保存图片}
利用QPixmap类获取图像和QFileDialog进行路径选择可以很容易的实现画板图像保存。

\begin{figure}[h] % [h] forces the figure to be output where it is defined in the code (it suppresses floating)
	\centering
	\includegraphics[width=0.6\columnwidth]{9.png} % Example image
	\caption{程序截图}
\end{figure}

\subsection{绘制曲线}
在绘制多边形的基础上利用QVector类存储各个顶点,通过鼠标左键选择各个采样点,最后通过右键点击通过QPainter的CubicTo方法插值生成平滑曲线

\pagebreak
\subsection{图标设置}
由于添加功能后工具栏文字很长,甚至一部分被自动隐藏,很不美观,故想到用图标代替各文字按钮。方法很简单,从网上下载相关图标
\begin{figure}[h] % [h] forces the figure to be output where it is defined in the code (it suppresses floating)
	\centering
	\includegraphics[width=0.6\columnwidth]{6.png} % Example image
	\caption{下载好的图标}
\end{figure}

对各个Action的定义中添加Icon:

\begin{figure}[h] % [h] forces the figure to be output where it is defined in the code (it suppresses floating)
	\centering
	\includegraphics[width=0.7\columnwidth]{7.png} % Example image
	\caption{程序截图}
\end{figure}

生成运行得到的效果:

\begin{figure}[h] % [h] forces the figure to be output where it is defined in the code (it suppresses floating)
	\centering
	\includegraphics[width=0.6\columnwidth]{8.png} % Example image
	\caption{实现效果}
\end{figure}

\pagebreak
\subsection{关于两个工具栏问题}

如下图所示,按照原框架生成的GUI会出现两个工具栏问题,这是由于在minidraw.ui中已经定义了一个默认的工具栏"mainToolBar",程序中使用addToolBar函数将会新建另外一个工具栏。

\begin{figure}[h] % [h] forces the figure to be output where it is defined in the code (it suppresses floating)
	\centering
	\includegraphics[width=0.6\columnwidth]{4.png} % Example image
	\caption{两个工具栏现象}
\end{figure}



解决该问题的方法是在minidraw.ui中删去ToolBar一项,在minidraw.h中包含头文件qtoolbar.h,即可解决问题,效果如下:

\begin{figure}[h] % [h] forces the figure to be output where it is defined in the code (it suppresses floating)
	\centering
	\includegraphics[width=0.6\columnwidth]{5.png} % Example image
	\caption{修正效果}
\end{figure}

\pagebreak


\section{实验效果}

\subsection{绘制直线}

\begin{figure}[h] % [h] forces the figure to be output where it is defined in the code (it suppresses floating)
	\centering
	\includegraphics[width=0.6\columnwidth]{12.png} % Example image
	\caption{绘制直线}
\end{figure}

\subsection{绘制矩形}

\begin{figure}[h] % [h] forces the figure to be output where it is defined in the code (it suppresses floating)
	\centering
	\includegraphics[width=0.6\columnwidth]{13.png} % Example image
	\caption{绘制矩形}
\end{figure}


\pagebreak
\subsection{绘制椭圆}

\begin{figure}[h] % [h] forces the figure to be output where it is defined in the code (it suppresses floating)
	\centering
	\includegraphics[width=0.6\columnwidth]{14.png} % Example image
	\caption{绘制直线}
\end{figure}

\subsection{自由绘图}

\begin{figure}[h] % [h] forces the figure to be output where it is defined in the code (it suppresses floating)
	\centering
	\includegraphics[width=0.6\columnwidth]{15.png} % Example image
	\caption{自由绘图}
\end{figure}

\pagebreak

\subsection{绘制多边形}

\begin{figure}[h] % [h] forces the figure to be output where it is defined in the code (it suppresses floating)
	\centering
	\includegraphics[width=0.6\columnwidth]{16.png} % Example image
	\caption{绘制多边形}
\end{figure}

\subsection{撤销操作}

\begin{figure}[h] % [h] forces the figure to be output where it is defined in the code (it suppresses floating)
	\begin{minipage}[t]{0.5\linewidth}
		\centering
		\includegraphics[width=0.9\columnwidth]{17.png}
		\caption{撤销前}
	\end{minipage}%
	\begin{minipage}[t]{0.5\linewidth}
		\centering
		\includegraphics[width=0.9 \columnwidth]{18.png}
		\caption{撤销后}
	\end{minipage}
\end{figure}

\pagebreak
\subsection{修改线色}

\begin{figure}[h] % [h] forces the figure to be output where it is defined in the code (it suppresses floating)
	\centering
	\includegraphics[width=1\columnwidth]{19.png} % Example image
	\caption{修改线色}
\end{figure}

\subsection{修改填充色}

\begin{figure}[h] % [h] forces the figure to be output where it is defined in the code (it suppresses floating)
	\centering
	\includegraphics[width=0.8\columnwidth]{20.png} % Example image
	\caption{修改填充色}
\end{figure}

\pagebreak
\subsection{修改线宽}

\begin{figure}[h] % [h] forces the figure to be output where it is defined in the code (it suppresses floating)
	\centering
	\includegraphics[width=0.8\columnwidth]{21.png} % Example image
	\caption{修改线宽}
\end{figure}

\subsection{保存图片}

\begin{figure}[h] % [h] forces the figure to be output where it is defined in the code (it suppresses floating)
	\centering
	\includegraphics[width=1\columnwidth]{22.png} % Example image
	\caption{保存图片}
\end{figure}

\pagebreak

\subsection{绘制曲线}

\begin{figure}[h] % [h] forces the figure to be output where it is defined in the code (it suppresses floating)
	\begin{minipage}[t]{0.5\linewidth}
		\centering
		\includegraphics[width=0.9\columnwidth]{23.png}
		\caption{选点}
	\end{minipage}%
	\begin{minipage}[t]{0.5\linewidth}
		\centering
		\includegraphics[width=0.9 \columnwidth]{24.png}
		\caption{画线}
	\end{minipage}
\end{figure}

\subsection{总结}

这次实验总体上难度不算大,因为一些电脑原先环境的原因,前期比较多的时间花在环境配置上,虽然之前没接触过Qt,对C++也不是很熟,但在理清框架的连接关系之后还是不禁赞叹面向对象的强大,原理理清之后各种功能无非就是拼凑衔接而已,期待后续实验。
%----------------------------------------------------------------------------------------

\end{document}
