%%%%%%%%%%%%%%%%%%%%%%%%%%%%%%%%%%%%%%%%%
% Wenneker Assignment
% LaTeX Template
% Version 2.0 (12/1/2019)
%
% This template originates from:
% http://www.LaTeXTemplates.com
%
% Authors:
% Vel (vel@LaTeXTemplates.com)
% Frits Wenneker
%
% License:
% CC BY-NC-SA 3.0 (http://creativecommons.org/licenses/by-nc-sa/3.0/)
% 
%%%%%%%%%%%%%%%%%%%%%%%%%%%%%%%%%%%%%%%%%

%----------------------------------------------------------------------------------------
%	PACKAGES AND OTHER DOCUMENT CONFIGURATIONS
%----------------------------------------------------------------------------------------

\documentclass[14pt]{scrartcl} % Font size

%%%%%%%%%%%%%%%%%%%%%%%%%%%%%%%%%%%%%%%%%
% Wenneker Assignment
% Structure Specification File
% Version 2.0 (12/1/2019)
%
% This template originates from:
% http://www.LaTeXTemplates.com
%
% Authors:
% Vel (vel@LaTeXTemplates.com)
% Frits Wenneker
%
% License:
% CC BY-NC-SA 3.0 (http://creativecommons.org/licenses/by-nc-sa/3.0/)
% 
%%%%%%%%%%%%%%%%%%%%%%%%%%%%%%%%%%%%%%%%%

%----------------------------------------------------------------------------------------
%	PACKAGES AND OTHER DOCUMENT CONFIGURATIONS
%----------------------------------------------------------------------------------------

\usepackage{amsmath, amsfonts, amsthm} % Math packages

\usepackage[utf8]{inputenc}

\usepackage[UTF8]{ctex}

\usepackage{listings} % Code listings, with syntax highlighting

\usepackage[english]{babel} % English language hyphenation

\usepackage{graphicx} % Required for inserting images
\graphicspath{{Figures/}{./}} % Specifies where to look for included images (trailing slash required)

\usepackage{booktabs} % Required for better horizontal rules in tables

\numberwithin{equation}{section} % Number equations within sections (i.e. 1.1, 1.2, 2.1, 2.2 instead of 1, 2, 3, 4)
\numberwithin{figure}{section} % Number figures within sections (i.e. 1.1, 1.2, 2.1, 2.2 instead of 1, 2, 3, 4)
\numberwithin{table}{section} % Number tables within sections (i.e. 1.1, 1.2, 2.1, 2.2 instead of 1, 2, 3, 4)

\setlength\parindent{0pt} % Removes all indentation from paragraphs

\usepackage{enumitem} % Required for list customisation
\setlist{noitemsep} % No spacing between list items

%----------------------------------------------------------------------------------------
%	DOCUMENT MARGINS
%----------------------------------------------------------------------------------------

\usepackage{geometry} % Required for adjusting page dimensions and margins

\geometry{
	paper=a4paper, % Paper size, change to letterpaper for US letter size
	top=2.5cm, % Top margin
	bottom=3cm, % Bottom margin
	left=3cm, % Left margin
	right=3cm, % Right margin
	headheight=0.75cm, % Header height
	footskip=1.5cm, % Space from the bottom margin to the baseline of the footer
	headsep=0.75cm, % Space from the top margin to the baseline of the header
	%showframe, % Uncomment to show how the type block is set on the page
}

%----------------------------------------------------------------------------------------
%	FONTS
%----------------------------------------------------------------------------------------

\usepackage[utf8]{inputenc} % Required for inputting international characters
\usepackage[T1]{fontenc} % Use 8-bit encoding

\usepackage{fourier} % Use the Adobe Utopia font for the document

%----------------------------------------------------------------------------------------
%	SECTION TITLES
%----------------------------------------------------------------------------------------

\usepackage{sectsty} % Allows customising section commands

\sectionfont{\vspace{6pt}\centering\normalfont\scshape} % \section{} styling
\subsectionfont{\normalfont\bfseries} % \subsection{} styling
\subsubsectionfont{\normalfont\itshape} % \subsubsection{} styling
\paragraphfont{\normalfont\scshape} % \paragraph{} styling

%----------------------------------------------------------------------------------------
%	HEADERS AND FOOTERS
%----------------------------------------------------------------------------------------

\usepackage{scrlayer-scrpage} % Required for customising headers and footers

\ohead*{} % Right header
\ihead*{} % Left header
\chead*{} % Centre header

\ofoot*{} % Right footer
\ifoot*{} % Left footer
\cfoot*{\pagemark} % Centre footer
 % Include the file specifying the document structure and custom commands

%----------------------------------------------------------------------------------------
%	TITLE SECTION
%----------------------------------------------------------------------------------------

\title{	
	\normalfont\normalsize
	%\vspace{25pt} % Whitespace
	\rule{\linewidth}{0.5pt}\\ % Thin top horizontal rule
	\vspace{20pt} % Whitespace
	{\huge 实验九	Path Tracing}\\ % The assignment title
	\vspace{12pt} % Whitespace
	\rule{\linewidth}{2pt}\\ % Thick bottom horizontal rule
	\vspace{12pt} % Whitespace
}

\author{\LARGE ID: 58		陈文博} % Your name

\date{\normalsize\today} % Today's date (\today) or a custom date

\begin{document}

\maketitle % Print the title

%----------------------------------------------------------------------------------------
%	FIGURE EXAMPLE
%----------------------------------------------------------------------------------------

\section{实验要求}

%\begin{figure}[h] % [h] forces the figure to be output where it is defined in the code (it suppresses floating)
%	\centering
%	\includegraphics[width=0.5\columnwidth]{swallow.jpg} % Example image
%	\caption{European swallow.}
%\end{figure}

\begin{itemize}
	\item[*] 实现路径追踪算法
	\item[*] 搭建场景并渲染
\end{itemize}

%------------------------------------------------

\section{开发环境}

\textbf{IDE}:Microsoft Visual Studio 2019 community\\
\textbf{CMake}:3.16.3\\
\textbf{Others}

%----------------------------------------------------------------------------------------
%	TEXT EXAMPLE
%----------------------------------------------------------------------------------------
\pagebreak
\section{算法原理}

\subsection{渲染方程}
\begin{equation}
L_o(\pmb{p},\pmb{\omega}_o)=L_e(\pmb{p},\pmb{\pmb{\omega}_o})+\int_{\mathcal{H}^2(\pmb{n}(\pmb{p}))} f_r(\pmb{p},\pmb{\omega}_i,\pmb{\omega}_o)L_i(\pmb{p},\pmb{\omega}_i)\cos\theta_{\pmb{\omega}_i,\pmb{n}(\pmb{p})}\mathbb{d}\pmb{\omega}_i
\end{equation}

参数释义:
\begin{itemize}
	\item $L_o$ 是出射 radiance
	\item $\pmb{p}$ 是渲染点
	\item $\pmb{\omega}_i$ 是入射光方向
	\item $\pmb{\omega}_o$ 是出射光方向
	\item $L_e$ 是发光 radiance
	\item $\pmb{n}(\pmb{p})$ 
	\item ${\mathcal{H}^2(\pmb{n}(\pmb{p}))} $ 是法向 $\pmb{n}(\pmb{p})$ 所在半球
	\item $f_r$ 是双向散射分布函数(BRDF)
	\item $L_i$ 是入射 radiance
	\item $\theta_{\pmb{\omega}_i,\pmb{n}(\pmb{p})}$ 是 $\pmb{\omega}_i$ 与 $\pmb{n}(\pmb{p})$ 的夹角
\end{itemize}
记
\begin{equation}
\int_{\pmb{p},\pmb{\omega}_o,\pmb{\omega}_i}L=\int_{\mathcal{H}^2(\pmb{n}(\pmb{p}))} f_r(\pmb{p},\pmb{\omega}_i,\pmb{\omega}_o)L\cos\theta_{\pmb{\omega}_i,\pmb{n}(\pmb{p})}\mathbb{d}\pmb{\omega}_i.
\end{equation}

则

\begin{equation}
L_o(\pmb{p},\pmb{\omega}_o)=L_e(\pmb{p},\pmb{\pmb{\omega}_o})+\int_{\pmb{p},\pmb{\omega}_o,\pmb{\omega}_i}L_i(\pmb{p},\pmb{\omega}_i)
\end{equation}

反射方程为

\begin{equation}
L_r(\pmb{p},\pmb{\omega}_o)=\int_{\pmb{p},\pmb{\omega}_o,\pmb{\omega}_i}L_i(\pmb{p},\pmb{\omega}_i)
\end{equation}

对于 $L_i$ 有关系式

\begin{equation}
L_i(\pmb{p},\pmb{\omega}_i)=L_o(\mathop{raytrace}(\pmb{p},\pmb{\omega_i}),-\pmb{\omega_i})
\end{equation}

其中,$raytrace$为射线与场景的相交函数

\pagebreak

记 $\mathop{raytrace}(\pmb{p},\pmb{\omega}_i)$ 为 $\pmb{p}^\prime$,则有

\begin{equation}
L_i(\pmb{p},\pmb{\omega}_i)=L_o(\pmb{p}^\prime,-\pmb{\omega_i})
\end{equation}


如此形成递归
\begin{equation}
L_o(\pmb{p},\pmb{\omega}_o)=L_e(\pmb{p},\pmb{\pmb{\omega}_o})+\int_{\pmb{p},\pmb{\omega}_o,\pmb{\omega}_i}L_o(\pmb{p}^\prime,-\pmb{\omega_i})
\end{equation}

将 $L_r$ 展开一次
\begin{equation}
\begin{aligned}
L_r(\pmb{p},\pmb{\omega}_o)
&=\int_{\pmb{p},\pmb{\omega}_o,\pmb{\omega}_i}\left(L_e(\pmb{p}^\prime,-\pmb{\omega}_i)+L_r(\pmb{p}^\prime,-\pmb{\omega}_i)\right)\\
&=\int_{\pmb{p},\pmb{\omega}_o,\pmb{\omega}_i}L_e(\pmb{p}^\prime,-\pmb{\omega}_i)
+\int_{\pmb{p},\pmb{\omega}_o,\pmb{\omega}_i}L_r(\pmb{p}^\prime,-\pmb{\omega}_i)
\end{aligned}
\end{equation}

记:
\begin{itemize}
	\item 直接光  $L_{\text{dir}}(\pmb{p},\pmb{\omega}_o)$:$\int_{\pmb{p},\pmb{\omega}_o,\pmb{\omega}_i}L_e(\pmb{p}^\prime,-\pmb{\omega}_i)$
	
	\item 间接光 $L_{\text{indir}}(\pmb{p},\pmb{\omega}_o)$  :$\int_{\pmb{p},\pmb{\omega}_o,\pmb{\omega}_i}L_r(\pmb{p}^\prime,-\pmb{\omega}_i)$
\end{itemize}
\subsection{直接光}
\begin{equation}
L_{\text{dir}}(\pmb{p},\pmb{\omega}_o)=\int_{\pmb{p},\pmb{\omega}_o,\pmb{\omega}_i}L_e(\pmb{p}^\prime,-\pmb{\omega}_i)
\end{equation}

积分中,对于大部分方向 $\pmb{\omega}_i$,$L_e(\pmb{p}^\prime,-\pmb{\omega}_i)=0$(非光源),所以我们直接在光源所在方向上积分

其中 $\pmb{p}$, $\pmb{\omega}_o$ 和 $\pmb{\omega}_i$ 可用三点确定,如下图所示

\begin{figure}[h] % [h] forces the figure to be output where it is defined in the code (it suppresses floating)
	\centering
	\includegraphics[width=0.3\columnwidth]{xyz.jpg} % Example image
	\caption{直接光反射图示}
\end{figure}

图中 $\pmb{x}$ 即为 $\pmb{p}$,$\pmb{y}$ 即为 $\pmb{p}^\prime$ 

\pagebreak

由几何关系可知

\begin{equation}
\mathbb{d}\pmb{\omega}_i=\frac{|\cos\theta_{\pmb{y},\pmb{x}}|}{\|\pmb{x}-\pmb{y}\|^2}\mathbb{d}A(\pmb{y})
\end{equation}

其中 $\theta_{\pmb{y},\pmb{x}}$ 是方向 $\pmb{x}-\pmb{y}$ 与 $\pmb{n}(\pmb{y})$ 的夹角

引入几何传输项(两点间的“传输效率”)
\begin{equation}
G(\pmb{x}\leftrightarrow\pmb{y})=V(\pmb{x}\leftrightarrow\pmb{y})\frac{|\cos\theta_{\pmb{x},\pmb{y}}||\cos\theta_{\pmb{y},\pmb{x}}|}{\|\pmb{x}-\pmb{y}\|^2}
\end{equation}

其中 $V(\pmb{x}\leftrightarrow\pmb{y})$ 是可见性函数,当 $\pmb{x}$ 和 $\pmb{y}$ 之间无阻隔时为 $1$,否则为 $0$ 

$G$ 是对称函数,即$G(\pmb{x}\leftrightarrow\pmb{y})=G(\pmb{y}\leftrightarrow\pmb{x})$ 

故有

$$
L_{\text{dir}}(\pmb{x}\to\pmb{z})=\int_A f_r(\pmb{y}\to \pmb{x}\to\pmb{z})L_e(\pmb{y}\to\pmb{x})G(\pmb{x}\leftrightarrow\pmb{y})\mathbb{d}A(\pmb{y})
$$

其中积分域 $A$ 为场景中所有的面积,但只有光源处 $L_e(\pmb{y}\to\pmb{x})\neq 0$ 

记光源数 $N_e$,场景中的光源集为 $\{L_{e_i}\}_{i=1}^{N_e}$ ,对应的区域集为 $\{A(L_{e_i})\}_{i=1}^{N_e}$,则可写为

$$
L_{\text{dir}}(\pmb{x}\to\pmb{z})=\sum_{i=1}^{N_e}\int_{A(L_{e_i})} f_r(\pmb{y}\to\pmb{x}\to\pmb{z})L_e(\pmb{y}\to\pmb{x})G(\pmb{x}\to\pmb{y})\mathbb{d}A(\pmb{y})
$$

\subsection{间接光}

递归求解:

\begin{equation}
L_r(\pmb{p},\pmb{\omega}_o)=\int_{\pmb{p},\pmb{\omega}_o,\pmb{\omega}_i}L_e(\pmb{p}^\prime,-\pmb{\omega}_i)
+\int_{\pmb{p},\pmb{\omega}_o,\pmb{\omega}_i}L_r(\pmb{p}^\prime,-\pmb{\omega}_i)
\end{equation}

\subsection{蒙特卡洛积分与重要性采样}

用采样的方法计算积分,方差与积分域$D$的维度无关

$X$为连续随机变量,$F$为随机变量函数:$F=g(X),X\sim p(x)$

$F$的期望:$E[F]=\int_Dg(x)p(x)dx$

$F$的估计:$F_N=\frac{1}{N}\sum\limits_{i=1}^Ng(x_i)\xrightarrow[]{N}E[Y]$

$f$与$g$的关系:$g(x)=\frac{f(x)}{g(x)}$

代入$F$的估计式:$\frac{1}{N}\sum\limits_{i=1}^N\frac{f(x_i)}{p(x_i)}\xrightarrow[]{N}\int_D f(x)dx$

方差:$V[F_N]=\frac{1}{N}V\Big[\frac{f(x)}{p(x)}\Big]\sim O(\frac{1}{N})$

为了缩小误差,除了增加样本数量,还可以缩小$V[\frac{f(x)}{p(x)}]$

\textbf{重要性采样}

若$p(x)=\frac{f(x)}{\int_D f(x)dx}$,则

\begin{equation}
V\Big[\frac{f(x)}{p(x)}\Big]=V\Big[\frac{1}{\int_D f(x)dx}\Big]
\end{equation}

只要$p(x)$和$f(x)$的形状接近,那么方差就会比较小,比如$f(x)=g(x)h(x)$,$h(x)\approx c$,而$g(x)$积分可求,则可取$p(x)=\frac{g(x)}{\int_D g(x)dx}$


\subsection{计算渲染方程}

我们要计算的是如下积分

\begin{equation}
L_r(\pmb{p},\pmb{\omega}_o)=L_{\text{dir}}+L_{\text{indir}}
\end{equation}

右侧积分式需要递归

利用蒙特卡洛积分可将积分转成采样

\begin{equation}
\begin{aligned}
L_{\text{dir}}(\pmb{x}\to\pmb{z})
&\approx\sum_{i=1}^{N_e}\sum_{j=1}^{N_i}\frac{f_r(\pmb{y}_i^{(j)}\to\pmb{x}\to\pmb{z})L_e(\pmb{y}_i^{(j)}\to\pmb{x})G(\pmb{x}\to\pmb{y}_i^{(j)})}{p(\pmb{y}_i^{(j)})}\\
L_{\text{indir}}(\pmb{p},\pmb{\omega}_o)
&\approx\sum_{k=1}^{N}\frac{f_r(\pmb{p},\pmb{\omega}_i^{(k)},\pmb{\omega}_o)L_r(\pmb{p}^{\prime(k)},-\pmb{\omega})\cos\theta_{\pmb{\omega}_i,\pmb{n}(\pmb{p})}}{p(\pmb{\omega}_i^{(k)})}
\end{aligned}
\end{equation}

$L_{\text{dir}}$ 在各光源区域采样

对于 $L_{\text{indir}}$ 则半球采样

采样个数皆为 1($N_i=1\quad(i=1,\dots,N_e)$,$N=1$) 

\pagebreak

\subsection{环境光贴图重要性采样}
\begin{figure}[h] % [h] forces the figure to be output where it is defined in the code (it suppresses floating)
	\centering
	\includegraphics[width=1.0\columnwidth]{is_em.jpg} % Example image
\end{figure}

\subsubsection{Alias Method}

\textbf{Operation}

算法利用一个概率表$U_i$和一个别名表$K_i$(for $1\leq i\leq n$),进行以下操作:

\begin{enumerate}
	\item 生成一个$[0,1)$的随机数$x$
	\item 设$i=\lfloor x \rfloor+1$和$y=nx+1-i$,使得$i\in \{1,2,\cdots,n\}$和$y\in[0,1)$
	\item 如果$y<U_i$,返回$i$
	\item 否则,返回$K_i$
\end{enumerate}

\textbf{Table Generation}
为了生成别名表,先初始化$U_i=np_i$

%------------------------------------------------
\pagebreak
\section{设计难点与解决}

\subsection{计算BRDF时入射光未正则化出现计算出错}

出错图像:
\begin{figure}[h] % [h] forces the figure to be output where it is defined in the code (it suppresses floating)
	\centering
	\includegraphics[width=0.7\columnwidth]{error.png} % Example image
	\caption{无正则化}
\end{figure}

\subsection{未考虑光源出射光方向}

面光源只朝一个方向出射光线,需要判断法向与返回光源的光线夹角排除从背部返回的光线,出错图像:

\begin{figure}[h] % [h] forces the figure to be output where it is defined in the code (it suppresses floating)
	\centering
	\includegraphics[width=0.7\columnwidth]{blur.png} % Example image
	\caption{算进从背部射入的光线会在光源附近产生一片亮区域}
\end{figure}

\pagebreak

\section{实验效果}

\subsection{标准场景测试}

\subsubsection{直接环境光}

\begin{figure}[h] % [h] forces the figure to be output where it is defined in the code (it suppresses floating)
	\centering
	\includegraphics[width=1.0\columnwidth]{env_dir.png} % Example image
\end{figure}

\subsubsection{直接+间接环境光}

\begin{figure}[h] % [h] forces the figure to be output where it is defined in the code (it suppresses floating)
	\centering
	\includegraphics[width=1.0\columnwidth]{env_indir.png} % Example image
\end{figure}

\pagebreak

\subsubsection{直接面光源}

\begin{figure}[h] % [h] forces the figure to be output where it is defined in the code (it suppresses floating)
	\centering
	\includegraphics[width=1.0\columnwidth]{area_dir.png} % Example image
\end{figure}

\subsubsection{直接+间接面光源}

\begin{figure}[h] % [h] forces the figure to be output where it is defined in the code (it suppresses floating)
	\centering
	\includegraphics[width=1.0\columnwidth]{area_indir.png} % Example image
\end{figure}

\pagebreak

\subsubsection{最终渲染结果128spp}

\begin{figure}[h] % [h] forces the figure to be output where it is defined in the code (it suppresses floating)
	\centering
	\includegraphics[width=1.0\columnwidth]{128spp.png} % Example image
\end{figure}

\subsubsection{最终渲染结果1024spp}

\begin{figure}[h] % [h] forces the figure to be output where it is defined in the code (it suppresses floating)
	\centering
	\includegraphics[width=1.0\columnwidth]{1024spp.png} % Example image
\end{figure}

\pagebreak

\subsection{其他渲染结果}

\subsubsection{室内}

\begin{figure}[h] % [h] forces the figure to be output where it is defined in the code (it suppresses floating)
	\centering
	\includegraphics[width=1.0\columnwidth]{ball.png} % Example image
\end{figure}

\subsubsection{戏剧院}
Square添加HW8中的校徽纹理贴图和法向贴图
\begin{figure}[h] % [h] forces the figure to be output where it is defined in the code (it suppresses floating)
	\centering
	\includegraphics[width=1.0\columnwidth]{ryg.png} % Example image
\end{figure}

\pagebreak

\subsubsection{海滨小镇}

\begin{figure}[h] % [h] forces the figure to be output where it is defined in the code (it suppresses floating)
	\centering
	\includegraphics[width=1.0\columnwidth]{sea.png} % Example image
\end{figure}

\subsubsection{雪地}

\begin{figure}[h] % [h] forces the figure to be output where it is defined in the code (it suppresses floating)
	\centering
	\includegraphics[width=1.0\columnwidth]{snow.png} % Example image
\end{figure}

\end{document}
