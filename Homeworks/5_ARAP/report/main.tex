%%%%%%%%%%%%%%%%%%%%%%%%%%%%%%%%%%%%%%%%%
% Wenneker Assignment
% LaTeX Template
% Version 2.0 (12/1/2019)
%
% This template originates from:
% http://www.LaTeXTemplates.com
%
% Authors:
% Vel (vel@LaTeXTemplates.com)
% Frits Wenneker
%
% License:
% CC BY-NC-SA 3.0 (http://creativecommons.org/licenses/by-nc-sa/3.0/)
% 
%%%%%%%%%%%%%%%%%%%%%%%%%%%%%%%%%%%%%%%%%

%----------------------------------------------------------------------------------------
%	PACKAGES AND OTHER DOCUMENT CONFIGURATIONS
%----------------------------------------------------------------------------------------

\documentclass[14pt]{scrartcl} % Font size

\input{structure.tex} % Include the file specifying the document structure and custom commands

%----------------------------------------------------------------------------------------
%	TITLE SECTION
%----------------------------------------------------------------------------------------

\title{	
	\normalfont\normalsize
	%\vspace{25pt} % Whitespace
	\rule{\linewidth}{0.5pt}\\ % Thin top horizontal rule
	\vspace{20pt} % Whitespace
	{\huge 实验五	\\ARAP Parameterization}\\ % The assignment title
	\vspace{12pt} % Whitespace
	\rule{\linewidth}{2pt}\\ % Thick bottom horizontal rule
	\vspace{12pt} % Whitespace
}

\author{\LARGE ID: 58		陈文博} % Your name

\date{\normalsize\today} % Today's date (\today) or a custom date

\begin{document}

\maketitle % Print the title

%----------------------------------------------------------------------------------------
%	FIGURE EXAMPLE
%----------------------------------------------------------------------------------------

\section{实验要求}

%\begin{figure}[h] % [h] forces the figure to be output where it is defined in the code (it suppresses floating)
%	\centering
%	\includegraphics[width=0.5\columnwidth]{swallow.jpg} % Example image
%	\caption{European swallow.}
%\end{figure}

\begin{itemize}
	\item[*] 在给定的网格框架上完成作业,实现
	\begin{itemize}
		\item ASAP (As-similar-as-possible) 参数化算法
		\item ARAP (As-rigid-as-possible) 参数化算法
		\item Hybrid 参数化方法(可选)
	\end{itemize}
	\item[*] 对各种参数化方法(包括作业4的Floater方法、ASAP/ARAP方法等)进行比较
	\item[*] 继续学习和巩固三角网格的数据结构及编程
	\item[*] 学习和实现矩阵的 SVD 分解
	\item[*] 进一步巩固使用 Eigen 库求解大型稀疏线性方程组
	

\end{itemize}

%------------------------------------------------

\section{开发环境}

\textbf{IDE}:Microsoft Visual Studio 2019 community\\
\textbf{CMake}:3.16.3\\
\textbf{Qt}:5.14.1\\
\textbf{Eigen}:3.3.7\\
\textbf{Assimp}:5.0.1\\
\textbf{tinyxml2}:8.0.0\\
\textbf{Others}


%----------------------------------------------------------------------------------------
%	TEXT EXAMPLE
%----------------------------------------------------------------------------------------
\pagebreak
\section{算法原理}

\subsection{基本方法}
非固定边界参数化基本步骤:将三维三角网格每个三角形全等映射到二维平面上,通过设计合适的算法,将每个三角形合并成二维网格(如下图所示)
\begin{figure}[h] % [h] forces the figure to be output where it is defined in the code (it suppresses floating)
	\centering
	\includegraphics[width=0.5\columnwidth]{bg.png} % Example image
	\caption{基本步骤 }
\end{figure}

寻找最佳映射,即求解以下能量函数:
\begin{equation}
E(u,L)=\sum\limits^T_{t=1}A_t\|J_t(u)-Lt\|^2_F
\end{equation}
其中$L_t$为各个三角形的线性变换,每个$L_t$都有相似的结构,
当变换为相似变换时,$L_t$有以下形式:

\begin{equation}
M=\left\{\begin{bmatrix}a&b\\-b&a\end{bmatrix}:a,b\in \boldsymbol R \right\}
\end{equation}

当变换为全等变换时,$L_t$有以下形式:

\begin{equation}
M=\left\{\begin{bmatrix}\cos\theta&\sin\theta\\-\sin\theta&\cos\theta\end{bmatrix}:\theta\in [0,2\pi) \right\}
\end{equation}

能量函数可重写为:
\begin{equation}
E(u,L)=\frac{1}{2}\sum\limits^T_{t=1}\sum\limits^2_{i=0}\cot(\theta_t^i)\|(u^i_t-u_t^{i+1})-L_t(x_t^i-x_t^{i+1})\|^2
\end{equation}

目标即求解最优化问题:
\begin{equation}
(u,L)=\arg\min_{(u,L)}E(u,L)\ \ s.t.\ \ L_t\in M
\end{equation}

\subsection{ASAP全局解法}
取$L_t$为:
\begin{equation}
M=\left\{\begin{bmatrix}a&b\\-b&a\end{bmatrix}:a,b\in \boldsymbol R \right\}
\end{equation}
此时$\theta_t^i$和$x_t^i$为已知量,可以以$a$,$b$和$u$作为未知量建立稀疏线性方程组进行优化问题求解

\subsection{Local/Global 方法}
\textbf{最优的$L$拟合矩阵}\\
利用SVD分解,可将$J$矩阵做以下分解:
\begin{equation}
J=U\Sigma V^T
\end{equation}

对于全等映射,$L$的拟合矩阵为:
\begin{equation}
L=UV^T
\end{equation}
即奇异值取1,当需要进行相似映射时,奇异值取其平均数$(\sigma_1+\sigma_2)/2$\\

\textbf{Local Phase}\\
利用二维三角形和当前估计坐标进行估计$J_t(u)$
\begin{equation}
J_t(u)\sim S_t(u)=\sum\limits_{t=0}^2 \cot(\theta_t^i)(u_t^i-u_t^{i+1})(x_t^i-x_t^{i+1})^T
\end{equation}

\textbf{Global Phase}\\
迭代求解新的估计坐标:
\begin{equation}
\begin{aligned}
&\sum\limits_{j\in N(t)}[\cot(\theta_{ij})+\cot(\theta_{ji}(u_i-u_j)\\
&=\sum\limits_{j\in N(t)}[\cot(\theta_{ij})L_{t(i,j)}+\cot(\theta_{ji}L_{t(j,i)})(x_i-x_j)\\
&\forall t=1,\cdots,n
\end{aligned}
)\end{equation}

\textbf{初始化}\\
使用Local/Global方法需要进行初始化,可采用固定边界的参数化作为初始参数坐标

\pagebreak
\section{设计难点与解决}

\subsection{关于ASAP全局解法的方程组构建}

一开始采用直接求导方法进行计算,由于在$x$和$y$方向都应求一次导加上$a$和$b$各为$T$维未知向量,直接求导将得到$2n+4T$个方程,$2n+2T$个未知量,系数矩阵非方阵,考虑求最小二乘解$AX=b\Leftrightarrow X=(A^TA)^{-1}A^Tb$,但由于情况复杂,计算结果未能调试成功,之后参考了文献\cite{liu2008local}中的附录B给出的ASAP方法下$a$,$b$和$u$的关系,即可实现将未知量数目减少至$2n$,简化了计算,由于时间受限暂未调试成功
%------------------------------------------------

\subsection{关于纹理映射问题}
一开始直接导入纹理坐标,发现纹理未能完全覆盖参数化平面,在网上查阅资料后发现可以通过修改OpenGL的参数实现纹理循环,而后在做ASAP的Local/Global方法时,由于特征值尺度不一导致展开的参数化平面大小随机(如下图)
\begin{figure}[h] % [h] forces the figure to be output where it is defined in the code (it suppresses floating)
	\centering
	\includegraphics[width=0.5\columnwidth]{not_normalize.png} % Example image
	\caption{参数化平面过大}
\end{figure}

后改为使用公式$\frac{x-x_{min}}{x_{max}-x_{min}}$进行归一化,将纹理坐标约束在0~1之间

\pagebreak

\section{实验效果}

\subsection{ARAP Local/Global方法}
\begin{table}[h] % [h] forces the table to be output where it is defined in the code (it suppresses floating)
	\centering % Centre the table
	\begin{tabular}{l l l l}
		\toprule
		\centering
		\textbf{原曲面} & \textbf{原网格} & \textbf{参数化网格} &\textbf{纹理映射}\\
		\midrule
		%%%%%%%%%%%%%%%%%%%%%%%%%%%%%%%%
		\begin{minipage}[t]{0.2\linewidth}
			\centering
			\includegraphics[width=0.9\linewidth]{ball.png}
		\end{minipage}&
		\begin{minipage}[t]{0.2\linewidth}
			\centering
			\includegraphics[width=0.9\linewidth]{ball_mesh.png}
		\end{minipage}&
		\begin{minipage}[t]{0.2\linewidth}
			\centering
			\includegraphics[width=0.9\linewidth]{ball_ARAP_3_mesh.png}
		\end{minipage}&
		\begin{minipage}[t]{0.2\linewidth}
			\centering
			\includegraphics[width=0.9\linewidth]{ball_ARAP_3.png}
		\end{minipage}\\
		%%%%%%%%%%%%%%%%%%%%%%%%%%%%%%%%%%%%%%%%%%%%%
		\begin{minipage}[t]{0.2\linewidth}
			\centering
			\includegraphics[width=0.9\linewidth]{bunny.png}
		\end{minipage}&
		\begin{minipage}[t]{0.2\linewidth}
			\centering
			\includegraphics[width=0.9\linewidth]{bunny_mesh.png}
		\end{minipage}&
		\begin{minipage}[t]{0.2\linewidth}
			\centering
			\includegraphics[width=0.9\linewidth]{bunny_ARAP_3_mesh.png}
		\end{minipage}&
		\begin{minipage}[t]{0.2\linewidth}
			\centering
			\includegraphics[width=0.8\linewidth]{bunny_ARAP_3.png}
		\end{minipage}\\
		
		%%%%%%%%%%%%%%%%%%%%%%%%%%%%%%%%%%%%%%%%%%%%%
		\begin{minipage}[t]{0.2\linewidth}
			\centering
			\includegraphics[width=0.9\linewidth]{cow.png}
		\end{minipage}&
		\begin{minipage}[t]{0.2\linewidth}
			\centering
			\includegraphics[width=0.9\linewidth]{cow_mesh.png}
		\end{minipage}&
		\begin{minipage}[t]{0.2\linewidth}
			\centering
			\includegraphics[width=0.8\linewidth]{cow_ARAP_3_mesh.png}
		\end{minipage}&
		\begin{minipage}[t]{0.2\linewidth}
			\centering
			\includegraphics[width=0.9\linewidth]{cow_ARAP_3.png}
		\end{minipage}\\
		
		%%%%%%%%%%%%%%%%%%%%%%%%%%%%%%%%%%%%%%%%%%%%%
		\begin{minipage}[t]{0.2\linewidth}
			\centering
			\includegraphics[width=0.9\linewidth]{grag.png}
		\end{minipage}&
		\begin{minipage}[t]{0.2\linewidth}
			\centering
			\includegraphics[width=0.8\linewidth]{grag_mesh.png}
		\end{minipage}&
		\begin{minipage}[t]{0.2\linewidth}
			\centering
			\includegraphics[width=0.9\linewidth]{garg_ARAP_3_mesh.png}
		\end{minipage}&
		\begin{minipage}[t]{0.2\linewidth}
			\centering
			\includegraphics[width=0.9\linewidth]{garg_ARAP_3.png}
		\end{minipage}\\
		
		%%%%%%%%%%%%%%%%%%%%%%%%%%%%%%%%%%%%%%%%%%%%%
		\begin{minipage}[t]{0.2\linewidth}
			\centering
			\includegraphics[width=0.9\linewidth]{iris.png}
		\end{minipage}&
		\begin{minipage}[t]{0.2\linewidth}
			\centering
			\includegraphics[width=0.9\linewidth]{iris_mesh.png}
		\end{minipage}&
		\begin{minipage}[t]{0.2\linewidth}
			\centering
			\includegraphics[width=0.9\linewidth]{iris_ARAP_3_mesh.png}
		\end{minipage}&
		\begin{minipage}[t]{0.2\linewidth}
			\centering
			\includegraphics[width=0.9\linewidth]{iris_ARAP_3.png}
		\end{minipage}\\
		
		
	\end{tabular}
	\caption{ARAP Local/Global方法 \\圆边界cotangent方法初始化 迭代3次}
\end{table}			
 
 
 \pagebreak
 
 \subsection{ASAP Local/Global方法}
 \begin{table}[h] % [h] forces the table to be output where it is defined in the code (it suppresses floating)
 	\centering % Centre the table
 	\begin{tabular}{l l l l}
 		\toprule
 		\centering
 		\textbf{原曲面} & \textbf{原网格} & \textbf{参数化网格} &\textbf{纹理映射}\\
 		\midrule
 		%%%%%%%%%%%%%%%%%%%%%%%%%%%%%%%%
 		\begin{minipage}[t]{0.2\linewidth}
 			\centering
 			\includegraphics[width=1.1\linewidth]{ball.png}
 		\end{minipage}&
 		\begin{minipage}[t]{0.2\linewidth}
 			\centering
 			\includegraphics[width=1.1\linewidth]{ball_mesh.png}
 		\end{minipage}&
 		\begin{minipage}[t]{0.2\linewidth}
 			\centering
 			\includegraphics[width=1.1\linewidth]{ball_ASAP_10_mesh.png}
 		\end{minipage}&
 		\begin{minipage}[t]{0.2\linewidth}
 			\centering
 			\includegraphics[width=1.1\linewidth]{ball_ASAP_10.png}
 		\end{minipage}\\
 		%%%%%%%%%%%%%%%%%%%%%%%%%%%%%%%%%%%%%%%%%%%%%
 		\begin{minipage}[t]{0.2\linewidth}
 			\centering
 			\includegraphics[width=1.1\linewidth]{bunny.png}
 		\end{minipage}&
 		\begin{minipage}[t]{0.2\linewidth}
 			\centering
 			\includegraphics[width=1.1\linewidth]{bunny_mesh.png}
 		\end{minipage}&
 		\begin{minipage}[t]{0.2\linewidth}
 			\centering
 			\includegraphics[width=1.1\linewidth]{bunny_ASAP_10_mesh.png}
 		\end{minipage}&
 		\begin{minipage}[t]{0.2\linewidth}
 			\centering
 			\includegraphics[width=1.1\linewidth]{bunny_ASAP_10.png}
 		\end{minipage}\\
 		
 		%%%%%%%%%%%%%%%%%%%%%%%%%%%%%%%%%%%%%%%%%%%%%
 		\begin{minipage}[t]{0.2\linewidth}
 			\centering
 			\includegraphics[width=1.1\linewidth]{grag.png}
 		\end{minipage}&
 		\begin{minipage}[t]{0.2\linewidth}
 			\centering
 			\includegraphics[width=1.1\linewidth]{grag_mesh.png}
 		\end{minipage}&
 		\begin{minipage}[t]{0.2\linewidth}
 			\centering
 			\includegraphics[width=1.1\linewidth]{garg_ASAP_10_mesh.png}
 		\end{minipage}&
 		\begin{minipage}[t]{0.2\linewidth}
 			\centering
 			\includegraphics[width=1.1\linewidth]{garg_ASAP_10.png}
 		\end{minipage}\\
 		
 		%%%%%%%%%%%%%%%%%%%%%%%%%%%%%%%%%%%%%%%%%%%%%
 		\begin{minipage}[t]{0.2\linewidth}
 			\centering
 			\includegraphics[width=1.1\linewidth]{iris.png}
 		\end{minipage}&
 		\begin{minipage}[t]{0.2\linewidth}
 			\centering
 			\includegraphics[width=1.1\linewidth]{iris_mesh.png}
 		\end{minipage}&
 		\begin{minipage}[t]{0.2\linewidth}
 			\centering
 			\includegraphics[width=1.1\linewidth]{iris_ASAP_10_mesh.png}
 		\end{minipage}&
 		\begin{minipage}[t]{0.2\linewidth}
 			\centering
 			\includegraphics[width=1.1\linewidth]{iris_ASAP_10.png}
 		\end{minipage}\\
 		
 		%%%%%%%%%%%%%%%%%%%%%%%%%%%%%%%%%%%%%%%%%%%%% 		
 		
 	\end{tabular}
 	\caption{ASAP Local/Global方法 圆边界cotangent方法初始化 \\迭代10次}
 \end{table}			


\pagebreak
 \subsubsection{ARAP迭代次数的影响}
\textbf{初始状态}
\begin{figure}[h] % [h] forces the figure to be output where it is defined in the code (it suppresses floating)
	\centering
	\includegraphics[width=0.5\columnwidth]{ball_circle_cotangent_mesh.png} % Example image
\end{figure}
\\
\textbf{迭代一次}
\begin{figure}[h] % [h] forces the figure to be output where it is defined in the code (it suppresses floating)
	\centering
	\includegraphics[width=0.5\columnwidth]{ball_ARAP_1_mesh.png} % Example image
\end{figure}
\pagebreak
\\
\textbf{迭代两次}
\begin{figure}[h] % [h] forces the figure to be output where it is defined in the code (it suppresses floating)
	\centering
	\includegraphics[width=0.5\columnwidth]{ball_ARAP_2_mesh.png} % Example image
	\end{figure}
\\
\textbf{迭代三次}
\begin{figure}[h] % [h] forces the figure to be output where it is defined in the code (it suppresses floating)
	\centering
	\includegraphics[width=0.5\columnwidth]{ball_ARAP_3_mesh.png} % Example image
\end{figure}
\pagebreak
\\
\textbf{迭代十次}
\begin{figure}[h] % [h] forces the figure to be output where it is defined in the code (it suppresses floating)
	\centering
	\includegraphics[width=0.5\columnwidth]{ball_ARAP_10_mesh.png} % Example image
\end{figure}
\\
\textbf{迭代一百次}
\begin{figure}[h] % [h] forces the figure to be output where it is defined in the code (it suppresses floating)
	\centering
	\includegraphics[width=0.5\columnwidth]{ball_ARAP_100_mesh.png} % Example image
\end{figure}
\\
可见ARAP的收敛是很快的
\pagebreak
\subsection{初始条件对ARAP的影响}
\textbf{cotangent权重,圆边界,迭代三次}
\begin{figure}[h] % [h] forces the figure to be output where it is defined in the code (it suppresses floating)
	\centering
	\includegraphics[width=0.5\columnwidth]{ball_ARAP_3_mesh.png} % Example image
\end{figure}
\\
\textbf{cotangent权重,正方形边界,迭代三次}
\begin{figure}[h] % [h] forces the figure to be output where it is defined in the code (it suppresses floating)
	\centering
	\includegraphics[width=0.5\columnwidth]{cotangent_square.png} % Example image
\end{figure}
\pagebreak
\\
\textbf{uniform权重,圆形边界,迭代三次}
\begin{figure}[h] % [h] forces the figure to be output where it is defined in the code (it suppresses floating)
	\centering
	\includegraphics[width=0.5\columnwidth]{uniform_circle.png} % Example image
\end{figure}
\\
\textbf{uniform权重,正方形形边界,迭代三次}
\begin{figure}[h] % [h] forces the figure to be output where it is defined in the code (it suppresses floating)
	\centering
	\includegraphics[width=0.5\columnwidth]{uniform_square.png} % Example image
\end{figure}
\\
可见使用不同初始化条件对收敛有一定影响,在本项中,cotangent权重要优于uniform权重,circle边界要由于square边界。

\pagebreak
\section{总结}
本次实验难度不小,主要是过程复杂调试困难,需要耐心和细心,由于前期过多时间花在构建ASAP方程组,导致一部分功能未能很好实现。

\bibliographystyle{unsrt}
\bibliography{bibfile}

\end{document}
