%%%%%%%%%%%%%%%%%%%%%%%%%%%%%%%%%%%%%%%%%
% Wenneker Assignment
% LaTeX Template
% Version 2.0 (12/1/2019)
%
% This template originates from:
% http://www.LaTeXTemplates.com
%
% Authors:
% Vel (vel@LaTeXTemplates.com)
% Frits Wenneker
%
% License:
% CC BY-NC-SA 3.0 (http://creativecommons.org/licenses/by-nc-sa/3.0/)
% 
%%%%%%%%%%%%%%%%%%%%%%%%%%%%%%%%%%%%%%%%%

%----------------------------------------------------------------------------------------
%	PACKAGES AND OTHER DOCUMENT CONFIGURATIONS
%----------------------------------------------------------------------------------------

\documentclass[14pt]{scrartcl} % Font size

%%%%%%%%%%%%%%%%%%%%%%%%%%%%%%%%%%%%%%%%%
% Wenneker Assignment
% Structure Specification File
% Version 2.0 (12/1/2019)
%
% This template originates from:
% http://www.LaTeXTemplates.com
%
% Authors:
% Vel (vel@LaTeXTemplates.com)
% Frits Wenneker
%
% License:
% CC BY-NC-SA 3.0 (http://creativecommons.org/licenses/by-nc-sa/3.0/)
% 
%%%%%%%%%%%%%%%%%%%%%%%%%%%%%%%%%%%%%%%%%

%----------------------------------------------------------------------------------------
%	PACKAGES AND OTHER DOCUMENT CONFIGURATIONS
%----------------------------------------------------------------------------------------

\usepackage{amsmath, amsfonts, amsthm} % Math packages

\usepackage[utf8]{inputenc}

\usepackage[UTF8]{ctex}

\usepackage{listings} % Code listings, with syntax highlighting

\usepackage[english]{babel} % English language hyphenation

\usepackage{graphicx} % Required for inserting images
\graphicspath{{Figures/}{./}} % Specifies where to look for included images (trailing slash required)

\usepackage{booktabs} % Required for better horizontal rules in tables

\numberwithin{equation}{section} % Number equations within sections (i.e. 1.1, 1.2, 2.1, 2.2 instead of 1, 2, 3, 4)
\numberwithin{figure}{section} % Number figures within sections (i.e. 1.1, 1.2, 2.1, 2.2 instead of 1, 2, 3, 4)
\numberwithin{table}{section} % Number tables within sections (i.e. 1.1, 1.2, 2.1, 2.2 instead of 1, 2, 3, 4)

\setlength\parindent{0pt} % Removes all indentation from paragraphs

\usepackage{enumitem} % Required for list customisation
\setlist{noitemsep} % No spacing between list items

%----------------------------------------------------------------------------------------
%	DOCUMENT MARGINS
%----------------------------------------------------------------------------------------

\usepackage{geometry} % Required for adjusting page dimensions and margins

\geometry{
	paper=a4paper, % Paper size, change to letterpaper for US letter size
	top=2.5cm, % Top margin
	bottom=3cm, % Bottom margin
	left=3cm, % Left margin
	right=3cm, % Right margin
	headheight=0.75cm, % Header height
	footskip=1.5cm, % Space from the bottom margin to the baseline of the footer
	headsep=0.75cm, % Space from the top margin to the baseline of the header
	%showframe, % Uncomment to show how the type block is set on the page
}

%----------------------------------------------------------------------------------------
%	FONTS
%----------------------------------------------------------------------------------------

\usepackage[utf8]{inputenc} % Required for inputting international characters
\usepackage[T1]{fontenc} % Use 8-bit encoding

\usepackage{fourier} % Use the Adobe Utopia font for the document

%----------------------------------------------------------------------------------------
%	SECTION TITLES
%----------------------------------------------------------------------------------------

\usepackage{sectsty} % Allows customising section commands

\sectionfont{\vspace{6pt}\centering\normalfont\scshape} % \section{} styling
\subsectionfont{\normalfont\bfseries} % \subsection{} styling
\subsubsectionfont{\normalfont\itshape} % \subsubsection{} styling
\paragraphfont{\normalfont\scshape} % \paragraph{} styling

%----------------------------------------------------------------------------------------
%	HEADERS AND FOOTERS
%----------------------------------------------------------------------------------------

\usepackage{scrlayer-scrpage} % Required for customising headers and footers

\ohead*{} % Right header
\ihead*{} % Left header
\chead*{} % Centre header

\ofoot*{} % Right footer
\ifoot*{} % Left footer
\cfoot*{\pagemark} % Centre footer
 % Include the file specifying the document structure and custom commands

%----------------------------------------------------------------------------------------
%	TITLE SECTION
%----------------------------------------------------------------------------------------

\title{	
	\normalfont\normalsize
	%\vspace{25pt} % Whitespace
	\rule{\linewidth}{0.5pt}\\ % Thin top horizontal rule
	\vspace{20pt} % Whitespace
	{\huge 实验七	Simulation Taichi}\\ % The assignment title
	\vspace{12pt} % Whitespace
	\rule{\linewidth}{2pt}\\ % Thick bottom horizontal rule
	\vspace{12pt} % Whitespace
}

\author{\LARGE ID: 58		陈文博} % Your name

\date{\normalsize\today} % Today's date (\today) or a custom date

\begin{document}

\maketitle % Print the title

%----------------------------------------------------------------------------------------
%	FIGURE EXAMPLE
%----------------------------------------------------------------------------------------

\section{实验要求}

%\begin{figure}[h] % [h] forces the figure to be output where it is defined in the code (it suppresses floating)
%	\centering
%	\includegraphics[width=0.5\columnwidth]{swallow.jpg} % Example image
%	\caption{European swallow.}
%\end{figure}

\begin{itemize}
	\item [*] 构建不同的模拟场景★
	\begin{itemize}
		\item 使用 `add\_objec` 函数设计自己的模拟场景(可以改变模拟物体的形状、粒子数目、初始位置、初始速度等等)
	\end{itemize}
	\item [*] 探究参数空间:通过调整不同的参数设置体会不同参数对仿真结果的影响★★
	\begin{itemize}
		\item 比如:探究杨氏模量E对雪模拟的影响,结合其背后的物理原理进行比较分析
	\end{itemize}
	\item [*] 探究其他算法和模型:通过改变物理模型、改变模拟算法实现新的模拟结果★★★
	\begin{itemize}
		\item 比如:水冲击弹性物体的模拟
	\end{itemize}
	\item [*] 渲染与绘制(可选 ★★★★★)
	\begin{itemize}
		\item 学习使用商业软件(如Houdini、Blender等)对结果进行渲染
		\item 【注:3D粒子更合适光照渲染,因此此处不作要求】
	\end{itemize}
\end{itemize}

%------------------------------------------------

\section{开发环境}

\textbf{IDE}:Microsoft Visual Studio 2019 community\\
\textbf{CMake}:3.16.3\\
\textbf{TaiChi} \\
\textbf{Others}

%----------------------------------------------------------------------------------------
%	TEXT EXAMPLE
%----------------------------------------------------------------------------------------
\section{算法原理}

\subsection{MLS-MPM算法流程\cite{hu2018mlsmpmcpic}}
\begin{figure}[h] % [h] forces the figure to be output where it is defined in the code (it suppresses floating)
	\centering
	\includegraphics[width=0.5\columnwidth]{process.png} % Example image
	\caption{算法流程}
\end{figure}
\pagebreak
\subsection{从粒子插值到格点}
\textbf{插值函数:}

\begin{equation}
N(x)=\begin{cases}
\frac{1}{2}|x|^3-x^2+\frac{2}{3}&0\leq |x|<1\\
-\frac{1}{6}|x|^3+x^2-2|x|+\frac{4}{3}&1\leq |x|<2\\
0&2\leq |x|

\end{cases}
\end{equation}

\textbf{B样条函数性质:}
\begin{equation}
\sum\limits_i w_{ip}=1
\end{equation}
\begin{equation}
\sum\limits_i w_{ip}\boldsymbol x_i=\boldsymbol x_p
\end{equation}

\textbf{格点质量计算:}\\
质量插值:
\begin{equation}
m_i^n=\sum\limits_p w_{ip}^nm_p^0
\end{equation}
质量守恒:
\begin{equation}
\sum\limits_i m_i^n=\sum\limits_i\sum\limits_p w_{ip}^nm_p^0=\sum\limits_p\left(\sum\limits_i w_{ip}^n\right)m_p^0=\sum\limits_p m_p^0
\end{equation}
\textbf{格点速度计算:}\\
动量插值:
\begin{equation}
m_i^n \boldsymbol v_i^n=\sum\limits_p w_{ip}^nm_p^0(\boldsymbol v_p^n+\boldsymbol C_p^{n+1}(x_i^n-x_i^p))
\end{equation}
\begin{equation}
\boldsymbol C_p=\boldsymbol B_p(\boldsymbol D_p)^{-1}
\end{equation}
\begin{equation}
\boldsymbol B_p=\sum\limits_iw_{ip}\boldsymbol v_i(x_i-x_p)^T
\end{equation}
\begin{equation}
\boldsymbol D_p=\sum\limits_i w_{ip}\boldsymbol v_i(x_i-x_p)(x_i-x_p)^T
\end{equation}
计算格点速度:
\begin{equation}
\boldsymbol v_i^n=\frac{m_i^n\boldsymbol v_i^n}{m_i^n}
\end{equation}
\pagebreak
\subsection{更新格点信息}
\textbf{格点受力计算}
\begin{equation}
\boldsymbol f_i^n=-\frac{\partial \Psi^n}{\partial \boldsymbol x_i^n}=-\sum\limits_p N_i(x_p^n)V_p^0M_p^{-1}\frac{\partial \psi_p^n(F_p^n)}{\partial \boldsymbol x_i^n}(x_i^n-x_i^p)
\end{equation}
\textbf{格点速度更新}
\begin{equation}
v_i^{n+1}=v_i^n+\frac{\boldsymbol f_i^n+\boldsymbol f^n_{iext}}{m_i^n}\Delta t
\end{equation}

\textbf{碰撞处理}
仅对在边界附近的格点进行处理

\subsection{从格点插值到粒子}
\textbf{粒子速度}
速度插值:
\begin{equation}
\boldsymbol v_p^{n+1}=\sum\limits_iw_{ip}^n\boldsymbol v_i^{n+1}
\end{equation}

\subsection{更新粒子信息}
\textbf{粒子位置更新}
\begin{equation}
\boldsymbol x_p^{n+1}=\boldsymbol x_p^n+\boldsymbol v_p^{n+1}\Delta t
\end{equation}
\textbf{形变梯度更新}
\begin{equation}
F_p^{n+1}=(I+\Delta t\boldsymbol C_p^{n+1})F_p^n
\end{equation}
%------------------------------------------------
\pagebreak
\section{仿真结果与分析}
由于仿真视频数量较多占空间较大,所有视频可以在网页\url{http://home.ustc.edu.cn/~chaf/material/CG/Taichi.html}上查看,或是通过百度云链接\url{https://pan.baidu.com/s/1Nzll2CPjqrpdODclYtoV8w}进行下载(提取码:6ens)
\subsection{模拟场景的构建}
\subsubsection{喷泉}
利用Sample中的喷水demo,修改喷水口大小和位置即可得到竖直向上的传统喷泉

\begin{figure}[h] % [h] forces the figure to be output where it is defined in the code (it suppresses floating)
	\centering
	\includegraphics[width=0.9\columnwidth]{fountain.png} % Example image
	\caption{喷泉}
\end{figure}
\pagebreak
\subsubsection{三彩喷泉}
添加三种不同颜色的喷泉,观察不同液体之间的交互过程
\begin{figure}[h]
	\centering
	\includegraphics[width=0.55\columnwidth]{3_fountain.png} % Example image
	\caption{开始不久}
	
	\centering
	\includegraphics[width=0.55\columnwidth]{3_fountain_1.png} % Example image
	\caption{一段时间后}
\end{figure}
\pagebreak
\subsubsection{不同材料的喷泉}
将喷泉材料设置为:液体、弹性物质、雪的材质
\begin{figure}[h]
	\centering
	\includegraphics[width=0.55\columnwidth]{material.png} % Example image
	\caption{开始不久}
	
	\centering
	\includegraphics[width=0.55\columnwidth]{material1.png} % Example image
	\caption{一段时间后}
\end{figure}

酷似水,绳子和牙膏

\pagebreak
\subsubsection{喷泉冲击弹性物体}
用四个喷泉从四个方向喷击方形弹性物体,观察不同物质之间的交互过程
\begin{figure}[h]
	\centering
	\includegraphics[width=0.55\columnwidth]{fountain_cube.png} % Example image
	\caption{开始不久}
	
	\centering
	\includegraphics[width=0.55\columnwidth]{fountain_cube_1.png} % Example image
	\caption{一段时间后}
\end{figure}

\pagebreak
\subsubsection{弹性马}
本来想实现PPT提供的Demo中的弹性马:
\begin{figure}[h]
	\centering
	\includegraphics[width=1\columnwidth]{horse.png} % Example image
	\caption{demo}
\end{figure}

未成功,做了个高仿的:
\begin{figure}[h]
	\centering
	\includegraphics[width=1\columnwidth]{horse_1.png} % Example image
	\caption{自己实现的}
\end{figure}

\pagebreak
\subsection{探索参数空间}
\subsubsection{杨氏模量E对弹性体碰撞性质的影响}
\begin{figure}[h]
	\centering
	\includegraphics[width=0.9\columnwidth]{Jelly_E.png} % Example image
	\caption{初始状态}
	
	\centering
	\includegraphics[width=0.9\columnwidth]{Jelly_E1.png} % Example image
	\caption{接触瞬间}
	
	\centering
	\includegraphics[width=0.9\columnwidth]{jelly_E2.png} % Example image
	\caption{弹起后}
\end{figure}

杨氏模量的公式为:$E=\sigma/\varepsilon$,其中,$E$表示杨氏模量,$\sigma$表示正向应力,$\varepsilon$表示正向应变。杨氏模量越大,在受到相同应力情况下,材料形变越小。从仿真结果可看出,在杨氏模量很小的情况下,弹性块直接被压扁了无法复原。在接触地面时,杨氏模量大的物块形变量小,更早的弹起.


\pagebreak
\subsubsection{杨氏模量E对雪块碰撞性质的影响}

\begin{figure}[h]
	\centering
	\includegraphics[width=1\columnwidth]{E1.png} % Example image
	\caption{初始状态}
	
	\centering
	\includegraphics[width=1\columnwidth]{E2.png} % Example image
	\caption{末状态}
\end{figure}

雪块碰撞地面不会发生弹起,当杨氏模量较小时,内部弹力很小,直接被压扁。当杨氏模量稍大时,雪块会碎开。继续增大杨氏模量,当雪块与地面发生撞击时会发生爆炸,这可能与仿真采样时间使得各粒子某时刻相聚过近导致应力急剧增大所致。

\pagebreak
\subsubsection{粒子数对不同材料物体仿真的影响}

实验使用同一个弹性小球以较大速度向下射击几块具有相同的较大体积、由不同材质、不同粒子数量构成的物体\\

\textbf{弹性材料}
\begin{figure}[h]
		\centering
	\includegraphics[width=0.9\columnwidth]{jelly_num2.png} % Example image
	\caption{射入后某个时间点}
\end{figure}

可以看出粒子数量越多,相同时间内可射入深度越小,当粒子数目为$5\times 10^5$时甚至接近无形变即刚体了。

\pagebreak
\begin{figure}[h]
	\centering
	\includegraphics[width=0.9\columnwidth]{jelly_num3.png} % Example image
	\caption{弹起后}
\end{figure}

当粒子数目为$5\times 10^2$和$5\times 10^3$时小球因速度过大在与地面接触后压扁,粒子数目为$5\times 10^4$时小球与物体粘连起来,这可能是接触瞬间速度过快导致瞬时粒子之间距离过小导致边界模糊,而当粒子数目为$5\times10^5$时,粒子重新弹起。当粒子能够重新弹起时,大体积物块会获得一部分能量导致振动等现象,因此小球能量会损耗

\pagebreak

\textbf{雪质材料}
\begin{figure}[h]
	\centering
	\includegraphics[width=0.9\columnwidth]{snow_num1.png} % Example image
	\caption{射入后某个时间点}
\end{figure}

当粒子数量多时,雪同样不易变形,对于粒子数少的情况,小球可以很容易地射到底部,而对于粒子数量很多的情况下已经开始有上升趋势

\pagebreak
\begin{figure}[h]
	\centering
	\includegraphics[width=0.9\columnwidth]{snow_num2.png} % Example image
	\caption{弹起后}
\end{figure}

粒子数少的情况下,小球轻易地射穿并重新弹起;粒子数量为$5\times 10^4$时,小球动能完全被吸收,转化为雪非弹性形变获得的内能(粒子微观动能);而当粒子数量为$5\times 10^5$时,雪的形变很少,小球内能损耗减少,故依旧可以重新弹起,此时雪和之前的弹性体一样,趋于刚体

\pagebreak
\textbf{流体材料}
\begin{figure}[h]
	\centering
	\includegraphics[width=0.9\columnwidth]{fluid_num1.png} % Example image
	\caption{射入后某个时间点}
\end{figure}

该时间点下,粒子数量$5\times 10^2$情况下小球以较大速度撞击地面压扁,粒子数$5\times10^3$时已重新弹起。结合后两种情况可以看出粒子数量越大,小球在流体中运动的能量损失越大,这是由于粒子数量越大,在流体与小球接触面内与小球接触的流体粒子数越多,小球与粒子相互碰撞能量交换更加频繁,能量损失越多

\pagebreak

\begin{figure}[htbp]
	\centering
	\includegraphics[width=0.9\columnwidth]{fluid_num2.png} % Example image
	\caption{末态}
\end{figure}

\begin{figure}[htbp]
	\centering
	\includegraphics[width=0.2\columnwidth]{water.png} % Example image
	\caption{现实情况}
\end{figure}

\pagebreak
\subsection{物理场景模拟}

\subsubsection{不同材质的球相互碰撞}
\begin{figure}[htbp]
	\centering
	\includegraphics[width=1\columnwidth]{ball1.png} % Example image
	\caption{开始时}
		\centering
	\includegraphics[width=1\columnwidth]{ball2.png} % Example image
	\caption{第一次碰撞后}
		\centering
	\includegraphics[width=1\columnwidth]{ball3.png} % Example image
	\caption{一段时间后}
\end{figure}

\pagebreak
\subsubsection{弹性小球撞击三种材质物体}
\begin{figure}[htbp]
	\centering
	\includegraphics[width=1\columnwidth]{hit1.png} % Example image
	\caption{开始时}
	\centering
	\includegraphics[width=1\columnwidth]{hit2.png} % Example image
	\caption{第一次碰撞后}
	\centering
	\includegraphics[width=1\columnwidth]{hit3.png} % Example image
	\caption{一段时间后}
\end{figure}

\pagebreak
\subsubsection{不同形状的弹性体射入流体中}
\begin{figure}[htbp]
	\centering
	\includegraphics[width=1.1\columnwidth]{shape1.png} % Example image
	\caption{开始时}
	
\end{figure}
\pagebreak
\begin{figure}[htbp]
	\centering
	\includegraphics[width=1.1\columnwidth]{shape2.png} % Example image
	\caption{接触后}
\end{figure}
\pagebreak
\begin{figure}[htbp]
	\centering
	\includegraphics[width=1.1\columnwidth]{shape3.png} % Example image
	\caption{一段时间后}
\end{figure}

可见物体与流体的接触面积越大,所受作用力越大,原本想用三角形代替一下“流线型”,然而事实上接触面更大了,随着角度的逐渐倾斜,阻力也越来越大,最终效果不如球和矩形

\pagebreak

\bibliographystyle{unsrt}
\bibliography{bibfile}

\end{document}
