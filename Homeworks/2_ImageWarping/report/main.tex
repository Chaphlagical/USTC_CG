%%%%%%%%%%%%%%%%%%%%%%%%%%%%%%%%%%%%%%%%%
% Wenneker Assignment
% LaTeX Template
% Version 2.0 (12/1/2019)
%
% This template originates from:
% http://www.LaTeXTemplates.com
%
% Authors:
% Vel (vel@LaTeXTemplates.com)
% Frits Wenneker
%
% License:
% CC BY-NC-SA 3.0 (http://creativecommons.org/licenses/by-nc-sa/3.0/)
% 
%%%%%%%%%%%%%%%%%%%%%%%%%%%%%%%%%%%%%%%%%

%----------------------------------------------------------------------------------------
%	PACKAGES AND OTHER DOCUMENT CONFIGURATIONS
%----------------------------------------------------------------------------------------

\documentclass[14pt]{scrartcl} % Font size

%%%%%%%%%%%%%%%%%%%%%%%%%%%%%%%%%%%%%%%%%
% Wenneker Assignment
% Structure Specification File
% Version 2.0 (12/1/2019)
%
% This template originates from:
% http://www.LaTeXTemplates.com
%
% Authors:
% Vel (vel@LaTeXTemplates.com)
% Frits Wenneker
%
% License:
% CC BY-NC-SA 3.0 (http://creativecommons.org/licenses/by-nc-sa/3.0/)
% 
%%%%%%%%%%%%%%%%%%%%%%%%%%%%%%%%%%%%%%%%%

%----------------------------------------------------------------------------------------
%	PACKAGES AND OTHER DOCUMENT CONFIGURATIONS
%----------------------------------------------------------------------------------------

\usepackage{amsmath, amsfonts, amsthm} % Math packages

\usepackage[utf8]{inputenc}

\usepackage[UTF8]{ctex}

\usepackage{listings} % Code listings, with syntax highlighting

\usepackage[english]{babel} % English language hyphenation

\usepackage{graphicx} % Required for inserting images
\graphicspath{{Figures/}{./}} % Specifies where to look for included images (trailing slash required)

\usepackage{booktabs} % Required for better horizontal rules in tables

\numberwithin{equation}{section} % Number equations within sections (i.e. 1.1, 1.2, 2.1, 2.2 instead of 1, 2, 3, 4)
\numberwithin{figure}{section} % Number figures within sections (i.e. 1.1, 1.2, 2.1, 2.2 instead of 1, 2, 3, 4)
\numberwithin{table}{section} % Number tables within sections (i.e. 1.1, 1.2, 2.1, 2.2 instead of 1, 2, 3, 4)

\setlength\parindent{0pt} % Removes all indentation from paragraphs

\usepackage{enumitem} % Required for list customisation
\setlist{noitemsep} % No spacing between list items

%----------------------------------------------------------------------------------------
%	DOCUMENT MARGINS
%----------------------------------------------------------------------------------------

\usepackage{geometry} % Required for adjusting page dimensions and margins

\geometry{
	paper=a4paper, % Paper size, change to letterpaper for US letter size
	top=2.5cm, % Top margin
	bottom=3cm, % Bottom margin
	left=3cm, % Left margin
	right=3cm, % Right margin
	headheight=0.75cm, % Header height
	footskip=1.5cm, % Space from the bottom margin to the baseline of the footer
	headsep=0.75cm, % Space from the top margin to the baseline of the header
	%showframe, % Uncomment to show how the type block is set on the page
}

%----------------------------------------------------------------------------------------
%	FONTS
%----------------------------------------------------------------------------------------

\usepackage[utf8]{inputenc} % Required for inputting international characters
\usepackage[T1]{fontenc} % Use 8-bit encoding

\usepackage{fourier} % Use the Adobe Utopia font for the document

%----------------------------------------------------------------------------------------
%	SECTION TITLES
%----------------------------------------------------------------------------------------

\usepackage{sectsty} % Allows customising section commands

\sectionfont{\vspace{6pt}\centering\normalfont\scshape} % \section{} styling
\subsectionfont{\normalfont\bfseries} % \subsection{} styling
\subsubsectionfont{\normalfont\itshape} % \subsubsection{} styling
\paragraphfont{\normalfont\scshape} % \paragraph{} styling

%----------------------------------------------------------------------------------------
%	HEADERS AND FOOTERS
%----------------------------------------------------------------------------------------

\usepackage{scrlayer-scrpage} % Required for customising headers and footers

\ohead*{} % Right header
\ihead*{} % Left header
\chead*{} % Centre header

\ofoot*{} % Right footer
\ifoot*{} % Left footer
\cfoot*{\pagemark} % Centre footer
 % Include the file specifying the document structure and custom commands

%----------------------------------------------------------------------------------------
%	TITLE SECTION
%----------------------------------------------------------------------------------------

\title{	
	\normalfont\normalsize
	%\vspace{25pt} % Whitespace
	\rule{\linewidth}{0.5pt}\\ % Thin top horizontal rule
	\vspace{20pt} % Whitespace
	{\huge 实验二	Image Warping}\\ % The assignment title
	\vspace{12pt} % Whitespace
	\rule{\linewidth}{2pt}\\ % Thick bottom horizontal rule
	\vspace{12pt} % Whitespace
}

\author{\LARGE ID: 58		陈文博} % Your name

\date{\normalsize\today} % Today's date (\today) or a custom date

\begin{document}

\maketitle % Print the title

%----------------------------------------------------------------------------------------
%	FIGURE EXAMPLE
%----------------------------------------------------------------------------------------

\section{实验要求}

%\begin{figure}[h] % [h] forces the figure to be output where it is defined in the code (it suppresses floating)
%	\centering
%	\includegraphics[width=0.5\columnwidth]{swallow.jpg} % Example image
%	\caption{European swallow.}
%\end{figure}

\begin{itemize}
	\item[*] 实现Image Warping 的两种算法
	\begin{itemize}
		\item Inverse distance-weighted interpolation method (IDW)
		\item Radial basis functions interpolation method (RBF)
	\end{itemize}
	\item[*] 对实验产生的白色裂缝进行填补
\end{itemize}

%------------------------------------------------

\section{开发环境}

\textbf{IDE}:Microsoft Visual Studio 2019 community\\
\textbf{CMake}:3.16.3\\
\textbf{Qt}:5.14.1\\
\textbf{Eigen}:3.3.7\\
\textbf{ANN}:1.1.2


%----------------------------------------------------------------------------------------
%	TEXT EXAMPLE
%----------------------------------------------------------------------------------------
\pagebreak
\section{算法原理}

\subsection{基本原理}
\begin{itemize}
	\item 输入:$n$对控制点对$(\boldsymbol{p_i},\boldsymbol{q_i}),i=1,2,\cdots,n$,其中$p_i\in \mathbb{R}^2$为控制源点,$q_i\in \mathbb{R}^2$为控制目标点
	\item 目标:找到一个映射$f:\mathbb{R}^2\rightarrow\mathbb{R}^2$,满足$f(\boldsymbol{p_i})=\boldsymbol{q_i},i=1,2,\cdots,n$
\end{itemize}

\subsection{Inverse distance-weighted interpolation methods(IDW)\cite{ruprecht1995image}}
IDW算法基本原理是根据给定的控制点对和控制点对的位移矢量,计算控制点对周围像素的反距离加权权重影响,实现图像每一个像素点的位移。\\
选择$n$组控制点对$(\boldsymbol{p_i},\boldsymbol{q_i}),i=1,2,\cdots,n$,目标映射$f:\mathbb{R}^2\rightarrow \mathbb{R}^2$可表示为以下形式:
\begin{equation}
f(\boldsymbol{p})=\sum^n_{i=1}\omega_i(\boldsymbol{p})f_i(\boldsymbol{p})
\end{equation}
其中,权重$w_i(\boldsymbol{p})$满足:
\begin{equation}
w_i(\boldsymbol{p})=\frac{\sigma_i(\boldsymbol{p})}{\sum\limits^n_{j=1} \sigma_j(\boldsymbol{p})}
\end{equation}
$\sigma_i(\boldsymbol{p})$反映第$i$对控制点对像素$\boldsymbol{p}$的反距离加权权重影响程度,可以直接取
\begin{equation}
\sigma_i(\boldsymbol{p})=\frac{1}{\|\boldsymbol{p}-\boldsymbol{p_i} \|^\mu} (\mu>1)
\end{equation}
也可以取locally bounded weight:
\begin{equation}
\sigma_i(\boldsymbol{p})=[\frac{(R_i-d(\boldsymbol{p},\boldsymbol{p_i}))}{R_id(\boldsymbol{p},\boldsymbol{p_i})}]^\mu
\end{equation}
$f_i$为线性函数,满足:
\begin{equation}
f_i(\boldsymbol{p})=\boldsymbol{q_i}+\boldsymbol{T_i}(\boldsymbol{p}-\boldsymbol{p_i})
\end{equation}
$\boldsymbol{T_i}$为二阶矩阵
\begin{equation}
\boldsymbol{T_i}=\begin{bmatrix}
t_{11}^{(i)}&t_{12}^{(i)}\\
t_{21}^{(i)}&t_{22}^{(i)}
\end{bmatrix}
\end{equation}
矩阵$\boldsymbol{T}$的确定,可通过求解最优化问题:
\begin{equation}
\min\limits_{\boldsymbol{T_i}} E_i(\boldsymbol{T_i})=\sum\limits^n_{j=1,j\neq i}\sigma_i(\boldsymbol{p_j})\|\boldsymbol{q_j}-f_i(\boldsymbol{p_j})\|^2
\end{equation}
对上式矩阵$\boldsymbol{T}$的各个元素分别求导,令方程为0可得
\begin{equation}
\boldsymbol{T}\sum\limits^n_{j=1,j\neq i}\sigma_i(\boldsymbol{p_j})\boldsymbol{\Delta p}\boldsymbol{\Delta p}^T =\sum\limits^n_{j=1,j\neq i}\sigma_i(\boldsymbol{p_j})\boldsymbol{\Delta q}\boldsymbol{\Delta p}^T
\end{equation}
其中$\boldsymbol{\Delta p}=(\boldsymbol{p_j}-\boldsymbol{p_i}\ \boldsymbol{0})$,$\boldsymbol{\Delta q}=(\boldsymbol{q_j}-\boldsymbol{q_i}\ \boldsymbol{0})$,当$\sum\limits^n_{j=1,j\neq i}\sigma_i(\boldsymbol{p_j})\boldsymbol{\Delta p}\boldsymbol{\Delta p}^T$非奇异时,可得
\begin{equation}
\boldsymbol{T}= (\sum\limits^n_{j=1,j\neq i}\sigma_i(\boldsymbol{p_j})\boldsymbol{\Delta q}\boldsymbol{\Delta p}^T)(\sum\limits^n_{j=1,j\neq i}\sigma_i(\boldsymbol{p_j})\boldsymbol{\Delta p}\boldsymbol{\Delta p}^T)^{-1}
\end{equation}
$\boldsymbol{T_i}(i=1,2,\cdots,n)$,确定以后映射$f$也就相应确定

\subsection{Radial basis functions interpolation method(RBF)\cite{arad1995image}}
选择$n$组控制点对$(\boldsymbol{p_i},\boldsymbol{q_i}),i=1,2,\cdots,n$,目标映射$f:\mathbb{R}^2\rightarrow \mathbb{R}^2$可表示为以下形式:
\begin{equation}
f(\boldsymbol{p})=\sum\limits_{i=1}^n\alpha_i g_i(\|\boldsymbol{p}-\boldsymbol{p_i}\|)+\boldsymbol{A}\boldsymbol{p}+\boldsymbol{B}
\end{equation}
其中,$g_i$是径向基函数,通常可取Hardy multiquadrics $g(t)=(t^2+c^2)^{\pm\frac{1}{2}}$或高斯函数$g_\sigma=e^{-t^2/\sigma^2}$,为了计算方便,这里取Hardy multiquadrics:
\begin{equation}
\begin{aligned}
g_i(d) & =(d+r_i)^{\pm\frac{1}{2}}\\
r_i & =\mathop{\min}\limits_{j\neq i} d(\boldsymbol{p_i},\boldsymbol{p_j})
\end{aligned}
\end{equation}
对于线性部分分量$\boldsymbol{A}\boldsymbol{p}+\boldsymbol{B}$,本例简单的取$\boldsymbol{A}=\boldsymbol{I}$和$\boldsymbol{B}=\boldsymbol{0}$


\pagebreak
\section{程序架构}
\subsection{文件结构}
\begin{figure}[h] % [h] forces the figure to be output where it is defined in the code (it suppresses floating)
	\centering
	\includegraphics[width=0.7\columnwidth]{1.png} % Example image
	\caption{文件结构}
\end{figure}


%------------------------------------------------
\pagebreak
\subsection{面向对象设计}

\begin{figure}[h] % [h] forces the figure to be output where it is defined in the code (it suppresses floating)
	\centering
	\includegraphics[width=1\columnwidth]{2.png} % Example image
	\caption{类图}
\end{figure}

在样例的基础上新增Warping父类,声明了初始化控制点、计算距离、像素点坐标变换、白缝填补的方法,子类WarpingRBF和WarpingIDW继承Warping,分别定义了各自的Image Warping算法。

%------------------------------------------------
\pagebreak
\section{设计难点与解决}

\subsection{图像位置的影响}

样例中的图像是默认处于窗口正中心,这会导致读取的控制点坐标系和图像内像素点坐标系不一致。需要在控制点初始化过程中,传入窗口大小信息对控制点对序列做平移变换预处理
%------------------------------------------------

\subsection{白缝的消除}

由于算法映射不完全,处理后图像会出现许些白缝,这里采用ANN库提供的邻域搜索,利用空像素周边的非空像素取均值进行插值。在配置ANN的过程中遇到不少困难,该库官方只提供Win32的链接库,而本例采用x64,无法正常链接。于是我利用Homework0中编译动态链接库的方法重新编译ANN源码,得到x64版本的链接库,经测试可正常使用。

\subsection{内存管理}

为了避免内存泄漏,用Qt附带的模板类QVector代替STL Vector,不需要显式释放内存,同时图像指针ptr\_image\_和ptr\_image\_backup\_需要在析构函数中进行释放,否则会发生内存泄漏。



\pagebreak

\section{实验效果}

\subsection{标准图像测试}
如下图所示,固定四角,蓝色为控制起始点,绿色为控制终止点
\begin{figure}[h] % [h] forces the figure to be output where it is defined in the code (it suppresses floating)
	\centering
	\includegraphics[width=0.5\columnwidth]{tag.png} % Example image
	\caption{拉伸情况}
\end{figure}

\subsection{IDW算法}

\subsubsection{$\mu=-1$情况}

\begin{figure}[h] % [h] forces the figure to be output where it is defined in the code (it suppresses floating)
	\begin{minipage}[t]{0.5\linewidth}
		\centering
		\includegraphics[width=0.8\linewidth]{IDW_mu=-1_nofix.jpg}
		\caption{修复前}
	\end{minipage}%
	\begin{minipage}[t]{0.5\linewidth}
		\centering
		\includegraphics[width=0.8 \linewidth]{IDW_mu=-1_fix.jpg}
		\caption{修复后}
	\end{minipage}
\end{figure}

\subsubsection{$\mu=1$情况}

\begin{figure}[h] % [h] forces the figure to be output where it is defined in the code (it suppresses floating)
	\begin{minipage}[t]{0.5\linewidth}
		\centering
		\includegraphics[width=0.8\linewidth]{IDW_mu=-2_nofix.jpg}
		\caption{修复前}
	\end{minipage}%
	\begin{minipage}[t]{0.5\linewidth}
		\centering
		\includegraphics[width=0.8 \linewidth]{IDW_mu=-2_fix.jpg}
		\caption{修复后}
	\end{minipage}
\end{figure}

\subsection{RBF算法}

\subsubsection{$\mu=0.5$情况}

\begin{figure}[h] % [h] forces the figure to be output where it is defined in the code (it suppresses floating)
	\begin{minipage}[t]{0.5\linewidth}
		\centering
		\includegraphics[width=0.8\linewidth]{RBF_mu=0_5_nofix.jpg}
		\caption{修复前}
	\end{minipage}%
	\begin{minipage}[t]{0.5\linewidth}
		\centering
		\includegraphics[width=0.8 \linewidth]{RBF_mu=0_5_fix.jpg}
		\caption{修复后}
	\end{minipage}
\end{figure}

\pagebreak
\subsubsection{$\mu=-0.5$情况}

\begin{figure}[h] % [h] forces the figure to be output where it is defined in the code (it suppresses floating)
	\begin{minipage}[t]{0.5\linewidth}
		\centering
		\includegraphics[width=0.8\linewidth]{RBF_mu=-0_5_nofix.jpg}
		\caption{修复前}
	\end{minipage}%
	\begin{minipage}[t]{0.5\linewidth}
		\centering
		\includegraphics[width=0.8 \linewidth]{RBF_mu=-0_5_fix.jpg}
		\caption{修复后}
	\end{minipage}
\end{figure}

\subsection{其他测试}

\subsubsection{柴犬表情包}
\textbf{原图片:}

\begin{figure}[h] % [h] forces the figure to be output where it is defined in the code (it suppresses floating)
	\centering
	\includegraphics[width=0.6\columnwidth]{dog.jpg} % Example image
	\caption{原始图片}
\end{figure}

\pagebreak
\textbf{处理后:}

\begin{figure}[h] % [h] forces the figure to be output where it is defined in the code (it suppresses floating)
	\begin{minipage}[t]{0.5\linewidth}
		\centering
		\includegraphics[width=0.9\linewidth]{dog1.jpg}
		\caption{Happy}
	\end{minipage}%
	\begin{minipage}[t]{0.5\linewidth}
		\centering
		\includegraphics[width=0.9 \linewidth]{dog2.bmp}
		\caption{Emmm...}
	\end{minipage}
\end{figure}

\subsubsection{平头哥也忍不住笑了}
\textbf{原图片:}

\begin{figure}[h] % [h] forces the figure to be output where it is defined in the code (it suppresses floating)
	\centering
	\includegraphics[width=0.9\columnwidth]{3.jpg} % Example image
	\caption{高冷}
\end{figure}

\pagebreak
\textbf{处理后:}

\begin{figure}[h] % [h] forces the figure to be output where it is defined in the code (it suppresses floating)
	\begin{minipage}[t]{0.5\linewidth}
		\centering
		\includegraphics[width=0.9\linewidth]{1.jpg}
		\caption{微笑}
	\end{minipage}%
	\begin{minipage}[t]{0.5\linewidth}
		\centering
		\includegraphics[width=0.9 \linewidth]{2.bmp}
		\caption{扭曲的微笑}
	\end{minipage}
\end{figure}

\section{总结}
本例中使用IDW和RBF两种方法进行图像的拉伸变换,理论上IDW和RBF的运算复杂度均为$O(n^2+nN)$,而由于实际运算中,IDW计算一个像素点的浮点乘法加法次数比RBF方法更多,在实验中也可以发现RBF处理速度要比IDW快3到4倍。

\bibliographystyle{unsrt}
\bibliography{bibfile}

\end{document}
