%%%%%%%%%%%%%%%%%%%%%%%%%%%%%%%%%%%%%%%%%
% Wenneker Assignment
% LaTeX Template
% Version 2.0 (12/1/2019)
%
% This template originates from:
% http://www.LaTeXTemplates.com
%
% Authors:
% Vel (vel@LaTeXTemplates.com)
% Frits Wenneker
%
% License:
% CC BY-NC-SA 3.0 (http://creativecommons.org/licenses/by-nc-sa/3.0/)
% 
%%%%%%%%%%%%%%%%%%%%%%%%%%%%%%%%%%%%%%%%%

%----------------------------------------------------------------------------------------
%	PACKAGES AND OTHER DOCUMENT CONFIGURATIONS
%----------------------------------------------------------------------------------------

\documentclass[14pt]{scrartcl} % Font size

%%%%%%%%%%%%%%%%%%%%%%%%%%%%%%%%%%%%%%%%%
% Wenneker Assignment
% Structure Specification File
% Version 2.0 (12/1/2019)
%
% This template originates from:
% http://www.LaTeXTemplates.com
%
% Authors:
% Vel (vel@LaTeXTemplates.com)
% Frits Wenneker
%
% License:
% CC BY-NC-SA 3.0 (http://creativecommons.org/licenses/by-nc-sa/3.0/)
% 
%%%%%%%%%%%%%%%%%%%%%%%%%%%%%%%%%%%%%%%%%

%----------------------------------------------------------------------------------------
%	PACKAGES AND OTHER DOCUMENT CONFIGURATIONS
%----------------------------------------------------------------------------------------

\usepackage{amsmath, amsfonts, amsthm} % Math packages

\usepackage[utf8]{inputenc}

\usepackage[UTF8]{ctex}

\usepackage{listings} % Code listings, with syntax highlighting

\usepackage[english]{babel} % English language hyphenation

\usepackage{graphicx} % Required for inserting images
\graphicspath{{Figures/}{./}} % Specifies where to look for included images (trailing slash required)

\usepackage{booktabs} % Required for better horizontal rules in tables

\numberwithin{equation}{section} % Number equations within sections (i.e. 1.1, 1.2, 2.1, 2.2 instead of 1, 2, 3, 4)
\numberwithin{figure}{section} % Number figures within sections (i.e. 1.1, 1.2, 2.1, 2.2 instead of 1, 2, 3, 4)
\numberwithin{table}{section} % Number tables within sections (i.e. 1.1, 1.2, 2.1, 2.2 instead of 1, 2, 3, 4)

\setlength\parindent{0pt} % Removes all indentation from paragraphs

\usepackage{enumitem} % Required for list customisation
\setlist{noitemsep} % No spacing between list items

%----------------------------------------------------------------------------------------
%	DOCUMENT MARGINS
%----------------------------------------------------------------------------------------

\usepackage{geometry} % Required for adjusting page dimensions and margins

\geometry{
	paper=a4paper, % Paper size, change to letterpaper for US letter size
	top=2.5cm, % Top margin
	bottom=3cm, % Bottom margin
	left=3cm, % Left margin
	right=3cm, % Right margin
	headheight=0.75cm, % Header height
	footskip=1.5cm, % Space from the bottom margin to the baseline of the footer
	headsep=0.75cm, % Space from the top margin to the baseline of the header
	%showframe, % Uncomment to show how the type block is set on the page
}

%----------------------------------------------------------------------------------------
%	FONTS
%----------------------------------------------------------------------------------------

\usepackage[utf8]{inputenc} % Required for inputting international characters
\usepackage[T1]{fontenc} % Use 8-bit encoding

\usepackage{fourier} % Use the Adobe Utopia font for the document

%----------------------------------------------------------------------------------------
%	SECTION TITLES
%----------------------------------------------------------------------------------------

\usepackage{sectsty} % Allows customising section commands

\sectionfont{\vspace{6pt}\centering\normalfont\scshape} % \section{} styling
\subsectionfont{\normalfont\bfseries} % \subsection{} styling
\subsubsectionfont{\normalfont\itshape} % \subsubsection{} styling
\paragraphfont{\normalfont\scshape} % \paragraph{} styling

%----------------------------------------------------------------------------------------
%	HEADERS AND FOOTERS
%----------------------------------------------------------------------------------------

\usepackage{scrlayer-scrpage} % Required for customising headers and footers

\ohead*{} % Right header
\ihead*{} % Left header
\chead*{} % Centre header

\ofoot*{} % Right footer
\ifoot*{} % Left footer
\cfoot*{\pagemark} % Centre footer
 % Include the file specifying the document structure and custom commands

%----------------------------------------------------------------------------------------
%	TITLE SECTION
%----------------------------------------------------------------------------------------

\title{	
	\normalfont\normalsize
	%\vspace{25pt} % Whitespace
	\rule{\linewidth}{0.5pt}\\ % Thin top horizontal rule
	\vspace{20pt} % Whitespace
	{\huge 实验八	Shader}\\ % The assignment title
	\vspace{12pt} % Whitespace
	\rule{\linewidth}{2pt}\\ % Thick bottom horizontal rule
	\vspace{12pt} % Whitespace
}

\author{\LARGE ID: 58		陈文博} % Your name

\date{\normalsize\today} % Today's date (\today) or a custom date

\begin{document}

\maketitle % Print the title

%----------------------------------------------------------------------------------------
%	FIGURE EXAMPLE
%----------------------------------------------------------------------------------------

\section{实验要求}

%\begin{figure}[h] % [h] forces the figure to be output where it is defined in the code (it suppresses floating)
%	\centering
%	\includegraphics[width=0.5\columnwidth]{swallow.jpg} % Example image
%	\caption{European swallow.}
%\end{figure}

\begin{itemize}
	\item[*] 实现 normal map 和 displacement map
	\item[*] 利用 displacement map 在 vertex shader 中进行简单降噪
	\item[*] 实现点光源阴影(可选)
\end{itemize}

%------------------------------------------------

\section{开发环境}

\textbf{IDE}:Microsoft Visual Studio 2019 community\\
\textbf{CMake}:3.16.3\\
\textbf{Assimp}:5.0.1\\
\textbf{GLFW}:3.4.0\\
\textbf{Others}

%----------------------------------------------------------------------------------------
%	TEXT EXAMPLE
%----------------------------------------------------------------------------------------
\pagebreak
\section{算法原理}

\subsection{法向贴图}

典型使用方式如下:

\begin{figure}[h] % [h] forces the figure to be output where it is defined in the code (it suppresses floating)
	\centering
	\includegraphics[width=1\columnwidth]{normal_map_usage.jpg} % Example image
	\caption{法向贴图的使用}
\end{figure}
由于贴图中的法向是定义在切空间中的,上方向为 z 方向,对应于 RGB 的 B 通道,故法线贴图一般为蓝紫色。根据表面细微的细节对法线向量进行改变,可以获得一种表面看起来要复杂得多的幻觉:
\begin{figure}[h] % [h] forces the figure to be output where it is defined in the code (it suppresses floating)
	\centering
	\includegraphics[width=1\columnwidth]{normal_mapping_surfaces.png} % Example image
	\caption{法向贴图的原理}
\end{figure}

\pagebreak
\subsection{置换贴图}
典型使用方式如下:
\begin{figure}[h] % [h] forces the figure to be output where it is defined in the code (it suppresses floating)
	\centering
	\includegraphics[width=1\columnwidth]{displacement_map_usage.jpg} % Example image
	\caption{置换贴图的使用}
\end{figure}

置换贴图中黑色(0)表示不动,白色(1)表示沿法向移动,可以定义置换贴图像素值与偏移量的关系:

displacement=lambda*(bias+scale*pixel value)
\begin{itemize}
	\item [*] bias=-1, scale=2: 0-0.5 为下沉,0.5-1.0 为凸起,变化量用一个系数决定
	\item [*] bias=0, scale=1: 0-1.0 为凸起,变化量用一个系数决定
\end{itemize}

在实时渲染中在vertex shader中采样置换贴图,计算顶点偏移量来偏移顶点,同时添加相应的法向贴图,使渲染效果更加正确。

\textbf{置换贴图用于降噪:}
\begin{itemize}
	\item 计算每个顶点的偏移量
	\begin{equation}
	\delta_i=p_i-\frac{1}{|N(i)|}\sum\limits_{j\in N(i)}p_j
	\end{equation}
	\item 将偏移量投影到法向上
	\begin{equation}
	\bar{\delta}_i=\langle\delta_i,\pmb{n}_i\rangle \pmb{n}_i
	\end{equation}
	\item 对每一个顶点进行偏移
	\begin{equation}
	\bar{p}_i=p_i-\lambda \bar{\delta}_i=p_i-\lambda\langle\delta_i,\pmb{n}_i\rangle \pmb{n}_i
	\end{equation}
	\item 将 $\langle\delta_i,\pmb{n}_i\rangle$ 插值存到置换贴图中
\end{itemize}


\pagebreak
\subsection{点光源阴影}
\subsubsection{从光源渲染出深度图}
\begin{figure}[h] % [h] forces the figure to be output where it is defined in the code (it suppresses floating)
	\centering
	\includegraphics[width=1\columnwidth]{render_from_light.png} % Example image
	\caption{从光源渲染}
\end{figure}

\subsubsection{从摄像机渲染}
\begin{figure}[h] % [h] forces the figure to be output where it is defined in the code (it suppresses floating)
	\centering
	\includegraphics[width=1\columnwidth]{render_from_camera.png} % Example image
	\caption{从摄像机渲染}
\end{figure}



%------------------------------------------------
\pagebreak
\section{设计难点与解决}

\subsection{用置换贴图去噪时displacementData插值}
使用Homework2中的ANN库进行k近邻插值,实际运行效果还行,但速度较慢

\subsection{置换贴图去噪时出现裂痕}
产生原因:由于网格拼接处相同位置的顶点采用相同的噪声,对其进行重心偏移后彼此分离造成裂痕。解决思路:遍历顶点寻找坐标相同点(衔接点)并保存,衔接点的偏移量设置为所有具有相同坐标的点偏移量的平均值。

\begin{figure}[h] % [h] forces the figure to be output where it is defined in the code (it suppresses floating)
	\centering
	\includegraphics[width=0.5\columnwidth]{crack.png} % Example image
	\caption{裂痕现象}
		\centering
	\includegraphics[width=0.5\columnwidth]{fix.png} % Example image
	\caption{裂痕修复}
\end{figure}

\pagebreak

\section{实验效果}
\subsection{法向贴图}
\begin{figure}[h] % [h] forces the figure to be output where it is defined in the code (it suppresses floating)
	\centering
	\includegraphics[width=0.8\columnwidth]{normal_map.png} % Example image
	\caption{法向贴图}
\end{figure}

\pagebreak
\subsection{置换贴图}
\begin{figure}[h] % [h] forces the figure to be output where it is defined in the code (it suppresses floating)
	\centering
	\includegraphics[width=0.6\columnwidth]{displacement_map.png} % Example image
	\caption{置换贴图}
\end{figure}

\textbf{其他效果:}
\begin{figure}[h] % [h] forces the figure to be output where it is defined in the code (it suppresses floating)
	\centering
	\includegraphics[width=0.6\columnwidth]{poem_map.png} % Example image
	\caption{台词贴图}
\end{figure}
\pagebreak
\begin{figure}[htbp] % [h] forces the figure to be output where it is defined in the code (it suppresses floating)
	\centering
	\includegraphics[width=0.6\columnwidth]{code_map.png} % Example image
	\caption{立体代码}
		\centering
	\includegraphics[width=0.6\columnwidth]{USTC_map.png} % Example image
	\caption{校徽贴图}
\end{figure}



\pagebreak
\subsection{降噪}
\begin{figure}[htbp] % [h] forces the figure to be output where it is defined in the code (it suppresses floating)
	\centering
	\includegraphics[width=0.6\columnwidth]{noise.png} % Example image
	\caption{带噪声的模型}
	
		\centering
	\includegraphics[width=0.6\columnwidth]{fix.png} % Example image
	\caption{降噪后的模型}
\end{figure}


\pagebreak

\end{document}
